\documentclass{article}

\usepackage{hyperref}
\usepackage{fancyhdr}
\usepackage{braket}
\usepackage{extramarks}
\usepackage{amsmath}
\usepackage{amsthm}
\usepackage{amsfonts}
\usepackage{tikz}
\usepackage[plain]{algorithm}
\usepackage{algpseudocode}
\usepackage{mathtools}
\usepackage{graphicx}
\graphicspath{ {./Images/} }

\DeclarePairedDelimiter\abs{\lvert}{\rvert}%
\DeclarePairedDelimiter\norm{\lVert}{\rVert}%

\makeatletter
\let\oldabs\abs
\def\abs{\@ifstar{\oldabs}{\oldabs*}}
%
\let\oldnorm\norm
\def\norm{\@ifstar{\oldnorm}{\oldnorm*}}
\makeatother

\newcommand*{\Value}{\frac{1}{2}x^2}%

\usetikzlibrary{automata,positioning}

%
% Basic Document Settings
%

\topmargin=-0.45in
\evensidemargin=0in
\oddsidemargin=0in
\textwidth=6.5in
\textheight=9.0in
\headsep=0.25in

\linespread{1.1}

\pagestyle{fancy}
\lhead{\hmwkAuthorName}
\chead{\hmwkClass\ (\hmwkClassInstructor\ \hmwkClassTime): \hmwkTitle}
\rhead{\firstxmark}
\lfoot{\lastxmark}
\cfoot{\thepage}

\renewcommand\headrulewidth{0.4pt}
\renewcommand\footrulewidth{0.4pt}

\setlength\parindent{0pt}

%
% Create Problem Sections
%

\newcommand{\enterProblemHeader}[1]{
    \nobreak\extramarks{}{Problem \arabic{#1} continued on next page\ldots}\nobreak{}
    \nobreak\extramarks{Problem \arabic{#1} (continued)}{Problem \arabic{#1} continued on next page\ldots}\nobreak{}
}

\newcommand{\exitProblemHeader}[1]{
    \nobreak\extramarks{Problem \arabic{#1} (continued)}{Problem \arabic{#1} continued on next page\ldots}\nobreak{}
    \stepcounter{#1}
    \nobreak\extramarks{Problem \arabic{#1}}{}\nobreak{}
}

\setcounter{secnumdepth}{0}
\newcounter{partCounter}
\newcounter{homeworkProblemCounter}
\setcounter{homeworkProblemCounter}{1}
\nobreak\extramarks{Problem \arabic{homeworkProblemCounter}}{}\nobreak{}

%
% Homework Problem Environment
%
% This environment takes an optional argument. When given, it will adjust the
% problem counter. This is useful for when the problems given for your
% assignment aren't sequential. See the last 3 problems of this template for an
% example.
%
\newenvironment{homeworkProblem}[1][-1]{
    \ifnum#1>0
        \setcounter{homeworkProblemCounter}{#1}
    \fi
    \section{Problem \arabic{homeworkProblemCounter}}
    \setcounter{partCounter}{1}
    \enterProblemHeader{homeworkProblemCounter}
}{
    \exitProblemHeader{homeworkProblemCounter}
}

%
% Homework Details
%   - Title
%   - Due date
%   - Class
%   - Section/Time
%   - Instructor
%   - Author
%

\newcommand{\hmwkTitle}{Homework\ \#11}
\newcommand{\hmwkDueDate}{April 14, 2020}
\newcommand{\hmwkClass}{Physics 926}
\newcommand{\hmwkClassTime}{}
\newcommand{\hmwkClassInstructor}{Professor Ken Bloom}
\newcommand{\hmwkAuthorName}{\textbf{Robert Tabb}}

%
% Title Page
%

\title{
    \vspace{2in}
    \textmd{\textbf{\hmwkClass:\ \hmwkTitle}}\\
    \normalsize\vspace{0.1in}\small{Due\ on\ \hmwkDueDate\ at 5pm}\\
    \vspace{0.1in}\large{\textit{\hmwkClassInstructor\ \hmwkClassTime}}
    \vspace{3in}
}

\author{\hmwkAuthorName}
\date{}

\renewcommand{\part}[1]{\textbf{\large Part \Alph{partCounter}}\stepcounter{partCounter}\\}

%
% Various Helper Commands
%

% Useful for algorithms
\newcommand{\alg}[1]{\textsc{\bfseries \footnotesize #1}}

% For derivatives
\newcommand{\deriv}[1]{\frac{\mathrm{d}}{\mathrm{d}x} (#1)}

% For partial derivatives
\newcommand{\pderiv}[2]{\frac{\partial}{\partial #1} (#2)}

% Integral dx
\newcommand{\dx}{\mathrm{d}x}

% Alias for the Solution section header
\newcommand{\solution}{\textbf{\large Solution}}

% Probability commands: Expectation, Variance, Covariance, Bias
\newcommand{\E}{\mathrm{E}}
\newcommand{\Var}{\mathrm{Var}}
\newcommand{\Cov}{\mathrm{Cov}}
\newcommand{\Bias}{\mathrm{Bias}}

\begin{document}

\maketitle

\pagebreak

\begin{homeworkProblem}
	Show that
	\[
		P(\nu_1 \rightarrow \nu_2) = \sin^22\theta\sin^2\frac{\Delta m_{12}^2L}{4E}
	\]
	for a two neutrino system in which the mixing matrix is
	\[
		U=\begin{pmatrix}
		\cos\theta & \sin\theta \\
		-\sin\theta & \cos\theta
		\end{pmatrix}
	\]
	and $\Delta m_{12}^2=m_1^2-m_2^2$
	\\
	\\
	\textbf{Solution}
	\\
	\\
	Equation (7) from the lecture gives us a good starting point. This equation is an approximation which is valid when mass is small, which in this case it is
	\[
		P(\nu_1 \rightarrow \nu_2) = \abs{\sum_i U_{1i}^* U_{2i} e^{-im_i^2L/2E}}^2
	\]
	Using this equation as a starting point, we can plug in the given values of $U$ and explicitly do the sum.
	\[
		\begin{split}
		\abs{\sum_i U_{1i}^* U_{2i} e^{-im_i^2L/2E}}^2 =& \left[ \sum_i U_{1i}^* U_{2i} e^{-im_i^2L/2E} \right] \left[ \sum_j U_{1j}^* U_{2j} e^{-im_j^2L/2E} \right]^* \\
		=& \left[ \sum_i U_{1i}^* U_{2i} e^{-im_i^2L/2E} \right] \left[ \sum_j U_{2j}^* U_{1j} e^{im_j^2L/2E} \right] \\
		\sum_i U_{1i}^* U_{2i} e^{-im_i^2L/2E} =& \cos\theta(-\sin\theta) e^{-im_1^2L/2E} + \sin\theta\cos\theta e^{-im_2^2L/2E} \\
		=& -\cos\theta\sin\theta e^{-im_1^2L/2E} + \cos\theta\sin\theta e^{-im_2^2L/2E} \\
		=& \cos\theta\sin\theta \left( e^{-im_2^2L/2E} - e^{-im_1^2L/2E} \right) \\
		\sum_j U_{2j}^* U_{1j} e^{im_j^2L/2E} =& -\sin\theta\cos\theta e^{im_1^2L/2E} + \cos\theta\sin\theta e^{im_2^2L/2E} \\
		=& \cos\theta\sin\theta \left( e^{im_2^2L/2E} - e^{im_1^2L/2E} \right) \\
		\end{split}
	\]
	\[
		\begin{split}
		\left[ \sum_i U_{1i}^* U_{2i} e^{-im_i^2L/2E} \right] \left[ \sum_j U_{2j}^* U_{1j} e^{im_j^2L/2E} \right] =& \cos^2\theta\sin^2\theta \left( e^{-im_2^2L/2E} - e^{-im_1^2L/2E} \right) \left( e^{im_2^2L/2E} - e^{im_1^2L/2E} \right) \\
		=& \cos^2\theta\sin^2\theta \left( 2 - e^{i(m_1^2-m_2^2)L/2E} - e^{-i(m_1^2-m_2^2)L/2E} \right) \\		
		=& \cos^2\theta\sin^2\theta \left( 2 - 2Re\left[e^{i(m_1^2-m_2^2)L/2E}\right] \right) \\
		=& 2\cos^2\theta\sin^2\theta \left( 1 - \cos\frac{\Delta m_{12}^2L}{2E}  \right)
		\end{split}
	\]
	From here, use the trig identity: \(1-cos\theta = 2\sin^2\frac{\theta}{2}\) and then $2\cos\theta\sin\theta = \sin2\theta$:
	\[
		\begin{split}
		\left[ \sum_i U_{1i}^* U_{2i} e^{-im_i^2L/2E} \right] \left[ \sum_j U_{2j}^* U_{1j} e^{im_j^2L/2E} \right] =&  2\cos^2\theta\sin^2\theta \left( 2\sin^2\frac{\Delta m_{12}^2L}{4E}  \right) \\
		=& 4\cos^2\theta\sin^2\theta \left( \sin^2\frac{\Delta m_{12}^2L}{4E}  \right) \\
		=& \sin^22\theta \sin^2\frac{\Delta m_{12}^2L}{4E}
		\end{split}
	\]
	

\end{homeworkProblem}

\pagebreak

\begin{homeworkProblem}
	As an exercise in natural units, show that the quantity $\Delta m_{12}^2L/4E$ that appears in the theory of neutrino oscillations is in fact equal to $1.27\Delta m_{12}^2(eV^2)L(km)/E(GeV)$.
	\\
	\\
	\textbf{Solution}
	\\
	\\
	In homework \#1, we had to derive conversion factors between natural units and SI units. In this problem we will use this to convert between length in $GeV^{-1}$ and $km$. The various ratios for conversions are:
	\[
		\begin{split}
		&L(GeV^{-1})=\frac{5.067\times 10^{15}\;GeV^{-1}}{1\;m}\frac{10^3 \; m}{1\; km}L(km) \\
		&\Delta m^2 (GeV^2) = \left( \frac{1\; GeV}{10^9\; eV} \right)^2 \Delta m^2 (eV^2)
		\end{split}
	\]
	Now plug these into the argument of the sine function from the previous problem to see what we get:
	\[
		\begin{split} 
		\frac{\Delta m^2 (GeV^2)L(GeV^{-1})}{4E(GeV)}=&\frac{\left( \frac{1\; GeV}{10^9\; eV} \right)^2 \Delta m^2 (eV^2)\frac{5.067\times 10^{15}\;GeV^{-1}}{1\;m}\frac{10^3 \; m}{1\; km}L(km)}{4E(GeV)} \\
		=& \frac{10^3}{10^{18}}\frac{5.067\times 10^{15}}{4}\frac{\Delta m^2 (eV^2)L(km)}{E (GeV)} \\
		=& 1.27 \frac{\Delta m^2 (eV^2)L(km)}{E (GeV)}
		\end{split}
	\]
\end{homeworkProblem}

\pagebreak

\begin{homeworkProblem}
	As mentioned in class, experiments such as NO$\nu$A are taking advantage of the fact that neutrinos that are traveling off-axis of a neutrino beam have a narrower energy spread. Let's take a look.
	\\
	\\
	(a) We want to make a neutrino beam from a beam of $\pi^+$ with $E_\pi = 20\; GeV$. How long should the decay pipe be to ensure the the great bulk of pions have decayed before they reach the absorber?
	\\
	\\
	(b) Consider a pion with energy $E_\pi$ in the laboratory frame. Find the energy of the neutrino $E_\nu$ in the decay $\pi^+\rightarrow \mu^+\nu_\mu$ as a function of the laboratory angle $\theta$ that the emitted neutrino makes with the original flight direction of the $\pi^+$.
	\\
	\\
	(c) Plot $E_\nu$ for $E_\pi$ between 2 and 20 $GeV$ in the case $\theta=0$ and $\theta=15 \; mrad$.
	\\
	\\
	\textbf{Solution}
	\\
	\\
	(a) Starting with the energy given, $E=20 \; GeV$, we can calculate velocity and time and use basic special relativity to get a good estimate. I am going to assume that the bulk of pions decaying means betwen 5 and 10 lifetimes (This may be a bit of an overestimate, but might as well err on the side of caution).
	\\
	\\
	Here are some relevant values:
	\[
		\begin{split}
		&\bar{\tau}=(2.603\pm 0.005)\times 10^{-8}s \\
		&m=0.14\; GeV = 1.78\times 10^{-27}\; kg \\
		&E=20\; GeV \\
		&p=\sqrt{E^2-m^2}\approx 20\;GeV = 1.07\times 10^{-17}\; kg\cdot m/s \\		
		\end{split}
	\]
	Where $\bar{\tau}$ is the mean lifetime, $m$ is the mass, $E$ is the energy, and $p$ is the momentum.\\
	Now I'll use these values to calculate velocity, lifetime in the rest frame of the particle, and thus the distance traveled.
	\[
		\begin{split}
		&p=\gamma mv=\frac{mv}{\sqrt{1-\frac{v^2}{c^2}}} \\
		&v=\frac{cp}{\sqrt{m^2c^2+\frac{p^2}{c^2}}} = \frac{(2.998\times 10^8)(1.07\times 10^{-17})}{\sqrt{(1.78\times 10^{-27}\cdot 2.998\times10^8)^2+\left(\frac{1.07\times 10^{-17}}{2.998\times 10^8}\right)^2}} \\
		&v=2.9979\times 10^8 m/s \approx 0.999976\; c \\
		&\gamma = \frac{1}{\sqrt{1-0.999976^2}} = 143.3 \\
		&\tau_{0,lower} = 5\gamma \bar{\tau} = 1.87\times 10^{-5}s \\
		&\tau_{0,upper} = 10\gamma \bar{\tau} = 3.73\times 10^{-5}s \\
		&d_{lower}=(2.9979\times 10^8)(1.87\times 10^{-5})  = 5.6\; km\\
		&d_{upper}=(2.9979\times 10^8)(3.73\times 10^{-5})  = 11.1\; km\\
		\end{split}
	\]
	Where $\tau_{0,lower}$ is the time estimate for five lifetimes, $\tau_{0,upper}$ is the time estimate for ten lifetimes, $d_{0,lower}$ is the distance using five lifetimes, and $d_{0,upper}$ is the distance using ten lifetimes. \\
	\\
	So based on my estimates the bulk of pions will be gone between $5.6$ and $11.1$ kilometers.
	\\
	\\
	(b) In the lab frame, the four-momenta of each particle can be defined. Let the initial direction of the pion be in the x-direction. ($\sigma$ is the index of the four-vectors while $\mu$ and $\nu$ are reserved for the muon and neutrino)
	\\
	\\
	Before the decay:
	\[
		P_\pi^\sigma = (E_\pi,p_\pi,0,0)
	\]
	After the decay:
	\[
		\begin{split}
		P_\mu^\sigma =& (E_\mu,\vec{p}_\mu) \\
		P_\nu^\sigma =& (E_\nu,p_\nu\cos\theta,p_\nu\sin\theta,0)
		\end{split}
	\]
	Note: I left the muon momentum completely general because it won't matter what value it has in the end.
	\\
	\\
	Due to conservation of four-momentum, we can write:
	\[
		\begin{split}
		P_\pi^\sigma = P_\mu^\sigma + P_\nu^\sigma \\
		P_\pi^\sigma - P_\nu^\sigma = P_\mu^\sigma \\		
		\end{split}
	\]
	We can contract each side with itself since this operation is Lorentz invariant:
	\[
		\begin{split}
		(P_\pi - P_\nu)^\sigma (P_\pi - P_\nu)_\sigma =& P_\mu^\sigma P_{\mu,\sigma} \\
		P_\pi^2+P_\nu^2-2P_\nu^\sigma P_{\pi,\sigma} =& P_\mu^2 \\
		m_\pi^2+m_\nu^2-2(E_\pi E_\nu-p_\pi p_\nu \cos\theta) =& m_\mu^2
		\end{split}
	\]
	$p_\nu \approx E_\nu$ because $m_\nu \approx 0$ compared with the other masses in the problem.
	\[
		\begin{split}		
		m_\pi^2-2(E_\pi E_\nu-m_\pi E_\nu \cos\theta) = m_\mu^2 \\
		m_\pi^2-m_\mu^2 = 2E_\nu(E_\pi-p_\pi \cos\theta) \\
		E_\nu = \frac{m_\pi^2-m_\mu^2}{2(E_\pi-p_\pi\cos\theta)} \\
		\end{split}
	\]
	But don't forget the dispersion relation:\(E^2=p^2+m^2\)
	\[
		\begin{split}
		\Rightarrow p_\pi = \sqrt{E_\pi^2-m_\pi^2}\\
		E_\nu = \frac{m_\pi^2-m_\mu^2}{2(E_\pi-\sqrt{E_\pi^2-m_\pi^2}\cos\theta)}
		\end{split}
	\]

	(c) Below is the plot of \(E_\nu = \frac{m_\pi^2-m_\mu^2}{2(E_\pi-\sqrt{E_\pi^2-m_\pi^2}\cos\theta)}\). 
	\begin{figure}[h]
		\includegraphics[scale=0.45]{prob3}
		\centering
		\caption{Neutrino energy as a function of pion energy for the decay, $\pi^+\rightarrow \nu_\mu \mu^+$}
		\label{prob3}
		\centering
	\end{figure}

\end{homeworkProblem}

\newpage

\begin{homeworkProblem} 
	We only briefly mentioned the possibility that neutrinos are their own antiparticles, i.e. are Majorana particles, and only briefly discussed the issue of CP violation. Let's explore these a little further. In class we said that for the three'generation mixing,
	\[
		P(\nu_\alpha \rightarrow \nu_\beta)= \delta_{\alpha\beta}-4\sum_{i>j}Re(U_{\alpha i}^*U_{\beta i}U_{\alpha j}U_{\beta j}^*)\sin^2\frac{\Delta m_{ij}^2L}{4E}+2\sum_{i>j}Im(U_{\alpha i}^*U_{\beta i}U_{\alpha j}U_{\beta j}^*)\sin^2\frac{\Delta m_{ij}^2L}{4E}
	\]
\end{homeworkProblem}
\end{document}

