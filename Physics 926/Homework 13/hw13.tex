\documentclass{article}

\usepackage{hyperref}
\usepackage{fancyhdr}
\usepackage{braket}
\usepackage{extramarks}
\usepackage{amsmath}
\usepackage{slashed}
\usepackage{amsthm}
\usepackage{amsfonts}
\usepackage{tikz}
\usepackage[plain]{algorithm}
\usepackage{algpseudocode}
\usepackage{mathtools}
\usepackage{graphicx}
\graphicspath{ {./Images/} }

\DeclarePairedDelimiter\abs{\lvert}{\rvert}%
\DeclarePairedDelimiter\norm{\lVert}{\rVert}%

\makeatletter
\let\oldabs\abs
\def\abs{\@ifstar{\oldabs}{\oldabs*}}
%
\let\oldnorm\norm
\def\norm{\@ifstar{\oldnorm}{\oldnorm*}}
\makeatother

\newcommand*{\Value}{\frac{1}{2}x^2}%

\usetikzlibrary{automata,positioning}

%
% Basic Document Settings
%

\topmargin=-0.45in
\evensidemargin=0in
\oddsidemargin=0in
\textwidth=6.5in
\textheight=9.0in
\headsep=0.25in

\linespread{1.1}

\pagestyle{fancy}
\lhead{\hmwkAuthorName}
\chead{\hmwkClass\ (\hmwkClassInstructor\ \hmwkClassTime): \hmwkTitle}
\rhead{\firstxmark}
\lfoot{\lastxmark}
\cfoot{\thepage}

\renewcommand\headrulewidth{0.4pt}
\renewcommand\footrulewidth{0.4pt}

\setlength\parindent{0pt}

%
% Create Problem Sections
%

\newcommand{\enterProblemHeader}[1]{
    \nobreak\extramarks{}{Problem \arabic{#1} continued on next page\ldots}\nobreak{}
    \nobreak\extramarks{Problem \arabic{#1} (continued)}{Problem \arabic{#1} continued on next page\ldots}\nobreak{}
}

\newcommand{\exitProblemHeader}[1]{
    \nobreak\extramarks{Problem \arabic{#1} (continued)}{Problem \arabic{#1} continued on next page\ldots}\nobreak{}
    \stepcounter{#1}
    \nobreak\extramarks{Problem \arabic{#1}}{}\nobreak{}
}

\setcounter{secnumdepth}{0}
\newcounter{partCounter}
\newcounter{homeworkProblemCounter}
\setcounter{homeworkProblemCounter}{1}
\nobreak\extramarks{Problem \arabic{homeworkProblemCounter}}{}\nobreak{}

%
% Homework Problem Environment
%
% This environment takes an optional argument. When given, it will adjust the
% problem counter. This is useful for when the problems given for your
% assignment aren't sequential. See the last 3 problems of this template for an
% example.
%
\newenvironment{homeworkProblem}[1][-1]{
    \ifnum#1>0
        \setcounter{homeworkProblemCounter}{#1}
    \fi
    \section{Problem \arabic{homeworkProblemCounter}}
    \setcounter{partCounter}{1}
    \enterProblemHeader{homeworkProblemCounter}
}{
    \exitProblemHeader{homeworkProblemCounter}
}

%
% Homework Details
%   - Title
%   - Due date
%   - Class
%   - Section/Time
%   - Instructor
%   - Author
%

\newcommand{\hmwkTitle}{Homework\ \#12}
\newcommand{\hmwkDueDate}{April 30, 2020}
\newcommand{\hmwkClass}{Physics 926}
\newcommand{\hmwkClassTime}{}
\newcommand{\hmwkClassInstructor}{Professor Ken Bloom}
\newcommand{\hmwkAuthorName}{\textbf{Robert Tabb}}

%
% Title Page
%

\title{
    \vspace{2in}
    \textmd{\textbf{\hmwkClass:\ \hmwkTitle}}\\
    \normalsize\vspace{0.1in}\small{Due\ on\ \hmwkDueDate\ at 5pm}\\
    \vspace{0.1in}\large{\textit{\hmwkClassInstructor\ \hmwkClassTime}}
    \vspace{3in}
}

\author{\hmwkAuthorName}
\date{}

\renewcommand{\part}[1]{\textbf{\large Part \Alph{partCounter}}\stepcounter{partCounter}\\}

%
% Various Helper Commands
%

% Useful for algorithms
\newcommand{\alg}[1]{\textsc{\bfseries \footnotesize #1}}

% For derivatives
\newcommand{\deriv}[1]{\frac{\mathrm{d}}{\mathrm{d}x} (#1)}

% For partial derivatives
\newcommand{\pderiv}[2]{\frac{\partial}{\partial #1} (#2)}

% Integral dx
\newcommand{\dx}{\mathrm{d}x}

% Alias for the Solution section header
\newcommand{\solution}{\textbf{\large Solution}}

% Probability commands: Expectation, Variance, Covariance, Bias
\newcommand{\E}{\mathrm{E}}
\newcommand{\Var}{\mathrm{Var}}
\newcommand{\Cov}{\mathrm{Cov}}
\newcommand{\Bias}{\mathrm{Bias}}

\begin{document}

\maketitle

\pagebreak

\begin{homeworkProblem}
	Suppose that instead of introducing an $SU(2)$ doublet of complex fields to do the symmetry breaking that generates the boson masses, we used an $SU(2)$ triplet instead. In that representation, the generators of the group are
	\[
		T^1=\frac{1}{\sqrt{2}}\begin{pmatrix} 0 & 1 & 0 \\ 1 & 0 & 1 \\ 0 & 1 & 0 \end{pmatrix} \;
		T^2=\frac{1}{\sqrt{2}}\begin{pmatrix} 0 & -i & 0 \\ i & 0 & -i \\ 0 & i & 0 \end{pmatrix} \;
		T^3=\begin{pmatrix} 1 & 0 & 0 \\ 0 & 0 & 0 \\ 0 & 0 & -1 \end{pmatrix}
	\]
	(a) Take 
	\[
		\phi_0=\begin{pmatrix} 0 \\ 0 \\ v \end{pmatrix}
	\]
	assign the hypercharge such that the field is electrically neutral. Calculate $M_W/M_Z$ in this model.
	\\
	\\
	(b) Now take
	\[
		\phi_0=\begin{pmatrix} 0 \\ v \\ 0 \end{pmatrix}
	\]
	and show that only the charged weak bosons acquire mass in this case.
	\\
	\\
	
	\textbf{Solution}
	\\
	\\
	(a) We want the field be be electrically neutral, this means that $Q\phi_0 = 0$
	\[
		\begin{split}
		Q\phi_0 =& \left(T_3+\frac{Y}{2}\right)\phi_0 \\
		=& T_3\phi_0+\frac{1}{2}Y\phi_0 \\
		=& -\phi_0+\frac{1}{2}Y\phi_0 \\
		=& 0
		\end{split}
	\]
	This means that a hypercharge of 2 gives us a field which is electrically neutral.
	\\
	\\
	Following from the lecture notes, we have the Lagrangian:
	\[
		\mathcal{L}=\abs{\left( i\partial_\mu-\frac{g}{2}\tau_aW_\mu^a-\frac{g'}{2}YB_\mu \right)\phi_0}^2-V\phi_0
	\]
	To find the masses, we only need the terms which are quadratic in the fields, and they come from the following piece, given by equation (24) from the lecture notes from 4/21:
	\[
		\begin{split}
		&\abs{\left(\frac{g}{2}\tau_aW_\mu^a+\frac{g'}{2}YB_\mu \right)\phi_0}^2 \\
		=& \frac{1}{4}\abs{ 
			g\begin{pmatrix} W_\mu^3 & \frac{1}{\sqrt2}(W_\mu^1-iW_\mu^2) & 0\\
							\frac{1}{\sqrt2}(W_\mu^1+iW_\mu^2)&0 & \frac{1}{\sqrt2}(W_\mu^1-iW_\mu^2)\\
							 0 & \frac{1}{\sqrt2}(W_\mu^1+iW_\mu^2) & -W_\mu^3 \end{pmatrix}
							 \begin{pmatrix} 0 \\ 0 \\ v \end{pmatrix}
							 +g'\begin{pmatrix} 2B_\mu & 0 & 0 \\ 0 & 2B_\mu & 0 \\ 0 & 0 & 2B_\mu
							  \end{pmatrix} \begin{pmatrix} 0 \\ 0 \\ v \end{pmatrix}}^2 \\
							=& \frac{1}{4}\abs{\begin{pmatrix} gW_\mu^3+2g'B_\mu & \frac{g}{\sqrt2}(W_\mu^1-iW_\mu^2) & 0\\
							\frac{g}{\sqrt2}(W_\mu^1+iW_\mu^2)& 2g'B_\mu & \frac{g}{\sqrt2}(W_\mu^1-iW_\mu^2)\\
							0 & \frac{g}{\sqrt2}(W_\mu^1+iW_\mu^2) & -gW_\mu^3+2g'B_\mu \end{pmatrix}\begin{pmatrix} 0 \\ 0 \\ v \end{pmatrix}}^2 \\
						=& \frac{1}{4}\begin{pmatrix} 0 & 0 & v \end{pmatrix}
						\begin{pmatrix} gW_\mu^3+2g'B_\mu & \frac{g}{\sqrt2}(W_\mu^1-iW_\mu^2) & 0\\
						\frac{g}{\sqrt2}(W_\mu^1+iW_\mu^2)& 2g'B_\mu & \frac{g}{\sqrt2}(W_\mu^1-iW_\mu^2)\\
						0 & \frac{g}{\sqrt2}(W_\mu^1+iW_\mu^2) & -gW_\mu^3+2g'B_\mu \end{pmatrix} \cdot
						\\
						\cdot& \begin{pmatrix} gW_\mu^3+2g'B_\mu & \frac{g}{\sqrt2}(W_\mu^1-iW_\mu^2) & 0\\
						\frac{g}{\sqrt2}(W_\mu^1+iW_\mu^2)& 2g'B_\mu & \frac{g}{\sqrt2}(W_\mu^1-iW_\mu^2)\\
						0 & \frac{g}{\sqrt2}(W_\mu^1+iW_\mu^2) & -gW_\mu^3+2g'B_\mu \end{pmatrix}
						\begin{pmatrix} 0 \\ 0 \\ v \end{pmatrix} \\
						=& \frac{1}{4}\begin{pmatrix} 0 & \frac{vg}{\sqrt{2}}(W_\mu^1+iW_\mu^2) & +v(-gW_\mu^3+2g'B_\mu) \end{pmatrix} 
						\begin{pmatrix} 0 \\ \frac{vg}{2}(W_\mu^1-iW_\mu^2) \\ v(-gW_\mu^3+2g'B_\mu) \end{pmatrix} \\
						=& \frac{v^2g^2}{8}\left[(W_\mu^1)^2+(W_\mu^2)^2\right]+\frac{v^2}{4}\left[-gW_\mu^3+2g'B_\mu\right]^2
		\end{split}
	\]
	What we see here is that we have two gauge bosons with the same mass such that \(m_W=\frac{vg}{2}\). These are the two w-bosons. Then we have another mass term which is a superposition of the $W$ and $B$ fields.
	\\
	\\
	Since we identify the mass terms in the Lagrangian in the form \(\frac{1}{2}m_G^2G^2\), where $G$ is some field and $m_G$ is its mass, we can associate this term with the second part of the above equation (since the photon mass is zero we know the only term we have must be the Z). We can now identify the Z-boson field (normalized):
	\[
		Z_\mu=\frac{2g'B_\mu-gW_\mu}{\sqrt{4g'^2+g^2}}
	\]
	\[
		\begin{split} 
		Z_\mu\sqrt{4g'^2+g^2}=2g'B_\mu-gW_\mu
		\end{split}
	\]
	Now compare the mass term with what was obtained above 
	\[
		\begin{split} 
		&\frac{v^2}{4}(2g'B_\mu-gW_\mu)^2=\frac{1}{2}\left[ \frac{v^2}{2}(4g'^2+g^2) \right]Z_\mu Z^\mu \\
		&\Rightarrow m_Z = v\sqrt{\frac{4g'^2+g^2}{2}}
		\end{split}
	\]
	From the text, we see that we can express these results in terms of $\theta_W$ (equations 15.22 and 15.23)
	\[
		\begin{split} 
		&Z_\mu=\cos\theta_WW_\mu^3 -\sin\theta_WB_\mu \\
		&\cos\theta_W = -g \\
		&\sin\theta_W = -2g' \\
		&\tan\theta_W = 2g'/g \\
		&g'=\frac{g\tan\theta_W}{2}
		\end{split}
	\]
	Let's use these identities to rewrite $m_Z$ in terms of just $g$ and then take the ratio of $m_W/m_Z$.
	\[
		\begin{split}
		m_Z=&v\sqrt{\frac{g^2\tan^2\theta_W+g^2}{2}} \\
		=&vg\sqrt{\frac{\tan^2\theta_W+1}{2}} \\
		=&vg\frac{\sec\theta_W}{\sqrt2} \\
		m_W/m_Z=& \frac{vg}{2} \frac{\sqrt2}{vg\sec\theta_W} \\
		=& \frac{\cos\theta_W}{\sqrt2}
		\end{split} 
	\]

	

\end{homeworkProblem}

\pagebreak

\begin{homeworkProblem} 
	
	\textbf{Solution}
	
\end{homeworkProblem} 

\pagebreak 

\begin{homeworkProblem} 
	Consider the spontaneous symmetry breaking of an $SU(2)$ local gauge symmetry, as we did in class. We made a particular choice of vacuum. Show that for any choice of vacuum, all three gauge bosons still acquire the same mass.
	\\
	\\
	\textbf{Solution}
	\\
	\\
	The kinetic energy term will wind up giving us the mass of the boson(s) as seen in equations (17) AND (18) from the lecture notes.
	\[
		\frac{g^2}{4}\left(\frac{1}{\sqrt{2}}\right)^2\abs{\begin{pmatrix} W_\mu^3 & W_\mu^1-iW_\mu^2 \\ W_\mu^1+iW_\mu^2 & -W_\mu^3 \end{pmatrix} \begin{pmatrix} 0 \\ v \end{pmatrix}}^2
	\]
	We can let $\phi$ be any generic choice and let the vector in the above equation be $\begin{pmatrix} v_1 \\ v_2 \end{pmatrix}$
	\[
		\begin{split} 
		&\frac{g^2}{4}\left(\frac{1}{\sqrt{2}}\right)^2\abs{\begin{pmatrix} W_\mu^3 & W_\mu^1-iW_\mu^2 \\ W_\mu^1+iW_\mu^2 & -W_\mu^3 \end{pmatrix}\begin{pmatrix} v_1 \\ v_2 \end{pmatrix} }^2 = \\
		=&\frac{g^2}{8}\left[ \begin{pmatrix} W_\mu^3 & W_\mu^1-iW_\mu^2 \\ W_\mu^1+iW_\mu^2 & -W_\mu^3 \end{pmatrix}\begin{pmatrix} v_1 \\ v_2 \end{pmatrix} \right]^\dagger \left[ \begin{pmatrix} W_\mu^3 & W_\mu^1-iW_\mu^2 \\ W_\mu^1+iW_\mu^2 & -W_\mu^3 \end{pmatrix}\begin{pmatrix} v_1 \\ v_2 \end{pmatrix} \right] \\
		=& \frac{g^2}{8} \left[ \begin{pmatrix} v_1 & v_2 \end{pmatrix} \begin{pmatrix} W_\mu^3 & W_\mu^1-iW_\mu^2 \\ W_\mu^1+iW_\mu^2 & -W_\mu^3 \end{pmatrix} \right]\left[ \begin{pmatrix} W_\mu^3 & W_\mu^1-iW_\mu^2 \\ W_\mu^1+iW_\mu^2 & -W_\mu^3 \end{pmatrix}\begin{pmatrix} v_1 \\ v_2 \end{pmatrix} \right] \\
		=& \frac{g^2}{8} \begin{pmatrix} W_\mu^3v_1+(W_\mu^1+iW_\mu^2)v_2 & (W_\mu^1-iW_\mu^2)v_1-W_\mu^3v_2 \end{pmatrix} 
		\begin{pmatrix} W_\mu^3v_1+(W_\mu^1-iW_\mu^2)v_2 \\ (W_\mu^1+iW_\mu^2)v_1-W_\mu^3v_2\end{pmatrix} \\
		=&\frac{g^2}{8} (W_\mu^3)^2v_1^2+(W_\mu^1+iW_\mu^2)(W_\mu^1-iW_\mu^2)v_2^2+W_\mu^3(W_\mu^1-iW_\mu^2)v_1v_2+W_\mu^3(W_\mu^1+iW_\mu^2)v_1v_2 \\
		+&\frac{g^2}{8}(W_\mu^1-iW_\mu^2)(W_\mu^1+iW_\mu^2)v_1^2+(W_\mu^3)^2v_2^2-W_\mu^3(W_\mu^1-iW_\mu^2)v_1v_2-W_\mu^3(W_\mu^1+iW_\mu^2)v_1v_2 \\
		=&\frac{g^2}{8} (W_\mu^3)^2v_1^2+(W_\mu^1)^2v_2^2+(W_\mu^2)^2v_2^2+  (W_\mu^3)^2v_2^2+(W_\mu^1)^2v_1^2+(W_\mu^2)^2v_1^2 \\
		=&\frac{g^2}{8} (v_1^2+v_2^2)(W_\mu^1)^2+(v_1^2+v_2^2)(W_\mu^2)^2+(v_1^2+v_2^2)(W_\mu^3)^2 \\
		=&\frac{g^2(v_1^2+v_2^2)}{8} \left[ (W_\mu^1)^2+(W_\mu^2)^2+(W_\mu^3)^2 \right]
		\end{split} 
	\]
	And now we have three gauge bosons with the same mass such that
	\[
		m=\frac{g\sqrt{v_1^2+v_2^2}}{2}
	\]
\end{homeworkProblem} 

\pagebreak 

\end{document}



