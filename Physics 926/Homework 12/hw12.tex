\documentclass{article}

\usepackage{hyperref}
\usepackage{fancyhdr}
\usepackage{braket}
\usepackage{extramarks}
\usepackage{amsmath}
\usepackage{slashed}
\usepackage{amsthm}
\usepackage{amsfonts}
\usepackage{tikz}
\usepackage[plain]{algorithm}
\usepackage{algpseudocode}
\usepackage{mathtools}
\usepackage{graphicx}
\graphicspath{ {./Images/} }

\DeclarePairedDelimiter\abs{\lvert}{\rvert}%
\DeclarePairedDelimiter\norm{\lVert}{\rVert}%

\makeatletter
\let\oldabs\abs
\def\abs{\@ifstar{\oldabs}{\oldabs*}}
%
\let\oldnorm\norm
\def\norm{\@ifstar{\oldnorm}{\oldnorm*}}
\makeatother

\newcommand*{\Value}{\frac{1}{2}x^2}%

\usetikzlibrary{automata,positioning}

%
% Basic Document Settings
%

\topmargin=-0.45in
\evensidemargin=0in
\oddsidemargin=0in
\textwidth=6.5in
\textheight=9.0in
\headsep=0.25in

\linespread{1.1}

\pagestyle{fancy}
\lhead{\hmwkAuthorName}
\chead{\hmwkClass\ (\hmwkClassInstructor\ \hmwkClassTime): \hmwkTitle}
\rhead{\firstxmark}
\lfoot{\lastxmark}
\cfoot{\thepage}

\renewcommand\headrulewidth{0.4pt}
\renewcommand\footrulewidth{0.4pt}

\setlength\parindent{0pt}

%
% Create Problem Sections
%

\newcommand{\enterProblemHeader}[1]{
    \nobreak\extramarks{}{Problem \arabic{#1} continued on next page\ldots}\nobreak{}
    \nobreak\extramarks{Problem \arabic{#1} (continued)}{Problem \arabic{#1} continued on next page\ldots}\nobreak{}
}

\newcommand{\exitProblemHeader}[1]{
    \nobreak\extramarks{Problem \arabic{#1} (continued)}{Problem \arabic{#1} continued on next page\ldots}\nobreak{}
    \stepcounter{#1}
    \nobreak\extramarks{Problem \arabic{#1}}{}\nobreak{}
}

\setcounter{secnumdepth}{0}
\newcounter{partCounter}
\newcounter{homeworkProblemCounter}
\setcounter{homeworkProblemCounter}{1}
\nobreak\extramarks{Problem \arabic{homeworkProblemCounter}}{}\nobreak{}

%
% Homework Problem Environment
%
% This environment takes an optional argument. When given, it will adjust the
% problem counter. This is useful for when the problems given for your
% assignment aren't sequential. See the last 3 problems of this template for an
% example.
%
\newenvironment{homeworkProblem}[1][-1]{
    \ifnum#1>0
        \setcounter{homeworkProblemCounter}{#1}
    \fi
    \section{Problem \arabic{homeworkProblemCounter}}
    \setcounter{partCounter}{1}
    \enterProblemHeader{homeworkProblemCounter}
}{
    \exitProblemHeader{homeworkProblemCounter}
}

%
% Homework Details
%   - Title
%   - Due date
%   - Class
%   - Section/Time
%   - Instructor
%   - Author
%

\newcommand{\hmwkTitle}{Homework\ \#12}
\newcommand{\hmwkDueDate}{April 21, 2020}
\newcommand{\hmwkClass}{Physics 926}
\newcommand{\hmwkClassTime}{}
\newcommand{\hmwkClassInstructor}{Professor Ken Bloom}
\newcommand{\hmwkAuthorName}{\textbf{Robert Tabb}}

%
% Title Page
%

\title{
    \vspace{2in}
    \textmd{\textbf{\hmwkClass:\ \hmwkTitle}}\\
    \normalsize\vspace{0.1in}\small{Due\ on\ \hmwkDueDate\ at 5pm}\\
    \vspace{0.1in}\large{\textit{\hmwkClassInstructor\ \hmwkClassTime}}
    \vspace{3in}
}

\author{\hmwkAuthorName}
\date{}

\renewcommand{\part}[1]{\textbf{\large Part \Alph{partCounter}}\stepcounter{partCounter}\\}

%
% Various Helper Commands
%

% Useful for algorithms
\newcommand{\alg}[1]{\textsc{\bfseries \footnotesize #1}}

% For derivatives
\newcommand{\deriv}[1]{\frac{\mathrm{d}}{\mathrm{d}x} (#1)}

% For partial derivatives
\newcommand{\pderiv}[2]{\frac{\partial}{\partial #1} (#2)}

% Integral dx
\newcommand{\dx}{\mathrm{d}x}

% Alias for the Solution section header
\newcommand{\solution}{\textbf{\large Solution}}

% Probability commands: Expectation, Variance, Covariance, Bias
\newcommand{\E}{\mathrm{E}}
\newcommand{\Var}{\mathrm{Var}}
\newcommand{\Cov}{\mathrm{Cov}}
\newcommand{\Bias}{\mathrm{Bias}}

\begin{document}

\maketitle

\pagebreak

\begin{homeworkProblem}
	If the vertex factor for the decay of a vector boson $X$ into two spin-1/2 fermions $f_1$ and $f_2$ is
	\[
		-igx\gamma^\mu\frac{1}{2}(c_v-c_A\gamma^5)
	\]
	then show that
	\[
		\Gamma(X\rightarrow f_1\bar{f}_2)=\frac{g_X^2}{48\pi}(c_v^2+c_A^2)M_X
	\]
	where $M_X$ is the mass of the boson and where we have neglected the mass of the fermions. Hints: use
	\[
		\sum_{\lambda} \epsilon_\mu^{(\lambda)*} \epsilon_\nu^\lambda=-g_{\mu\nu}+\frac{p_\mu p_\nu}{M^2}
	\]
	to show that after summing over the fermions and averaging over the boson spins ,
	\[
		\overline{\abs{\mathcal{M}}^2}=\frac{1}{12} g_X^2(c_v^2+c_A^2)(-g_{\mu\nu})Tr(\gamma^\mu\slashed{k}\gamma^\nu\slashed{k}')
	\]
	where $k$ and $k'$ are the four-momenta of the fermions. Work in the boson rest frame, and use
	\[
		\Gamma(X\rightarrow f_1\bar{f}_2)\frac{p_f}{32\pi^2m_X^2}\int \overline{\abs{\mathcal{M}}^2} d\Omega
	\]
\textbf{Solution}
\\
\\
	Using the vertex factor given in the problem, begin by writing the matrix element
	\[
		\begin{split}
		\mathcal{M}=&\bar{u}(k)\left[-ig_X\gamma^\mu\frac{1}{2}(c_v-c_A\gamma^5)\right]v(k')\epsilon_\mu \\
		=&-\frac{ig_X}{2}\left[\bar{u}(k)\gamma^\mu(c_v-c_A\gamma^5)v(k')\epsilon_\mu\right] \\
		\abs{\mathcal{M}}^2 =&\frac{g_X^2}{4}\left[\bar{u}(k)\gamma^\mu(c_v-c_A\gamma^5)v(k')\epsilon_\mu\right]\left[\bar{u}(k)\gamma^\nu(c_v-c_A\gamma^5)v(k')\epsilon_\nu\right]^*
		\end{split}
	\]
	We now need to write the average by summing over the spins and polarizations. In this step, assume the fermion masses can be neglected and use the formula: \( \sum_s [\bar{u}(a)\Gamma_1v(b)][\bar{u}(a)\Gamma_1v(b)]^* = Tr[\Gamma_1\slashed{b}\bar{\Gamma}_2\slashed{a}]  \), recalling that \( \bar{\Gamma} =\gamma^0\Gamma^\dagger\gamma^0 \)
	\[
		\begin{split}
		\overline{\abs{\mathcal{M}}^2} =& \frac{1}{3}\sum_{s,\lambda}\abs{\mathcal{M}}^2 \\
		=&\frac{g_X^2}{12}\sum_\lambda\left( \epsilon_\mu \epsilon_\nu^* \right) \sum_s \left[\bar{u}(k)\gamma^\mu(c_v-c_A\gamma^5)v(k')\right]\left[\bar{u}(k)\gamma^\nu(c_v-c_A\gamma^5)v(k')\right]^* \\
		=&\frac{g_X^2}{12} \left( -g_{\mu\nu}+\frac{q_\mu q_\nu}{M_X^2} \right)Tr\left[ \gamma^\mu(c_v-c_A\gamma^5)\slashed{k}'\gamma^0(\gamma^\nu(c_v-c_A\gamma^5))^\dagger\gamma^0\slashed{k} \right] \\
		=& \frac{g_X^2}{12} \left( -g_{\mu\nu}+\frac{q_\mu q_\nu}{M_X^2} \right)Tr\left[ (c_v\gamma^\mu\slashed{k}'-c_A\gamma^\mu\gamma^5\slashed{k}') (c_v\gamma^0\gamma^{\nu\dagger}\gamma^0\slashed{k}-c_A\gamma^0\gamma^{5\dagger}\gamma^{\nu\dagger}\gamma^0\slashed{k}) \right] \\
		=& \frac{g_X^2}{12} \left( -g_{\mu\nu}+\frac{q_\mu q_\nu}{M_X^2} \right) Tr\left[ (c_v\gamma^\mu\slashed{k}'-c_A\gamma^\mu\gamma^5\slashed{k}')    (c_v\gamma^{\nu}\slashed{k}+c_A\gamma^{5}\gamma^{\nu}\slashed{k}) \right] \\
		\end{split}
	\]
	Where the plus sign in the last line comes from letting \(\gamma^0\gamma^5\rightarrow \gamma^5\gamma^0  \)
	Now let's evaluate that trace
	\[
		\begin{split} 
		Tr\left[(c_v\gamma^\mu\slashed{k}'-c_A\gamma^\mu\gamma^5\slashed{k}')    (c_v\gamma^{\nu}\slashed{k}+c_A\gamma^{5}\gamma^{\nu}\slashed{k})  \right] =& Tr\left[ c_v^2\gamma^\mu\slashed{k}'\gamma^{\nu}\slashed{k}-c_A^2\gamma^\mu\gamma^5\slashed{k}'\gamma^{5}\gamma^{\nu}\slashed{k} -2c_vc_A\gamma^\mu\gamma^5\slashed{k}'\gamma^\nu\slashed{k} \right] \\
		=& c_v^2Tr\left[ \gamma^\mu\slashed{k}'\gamma^{\nu}\slashed{k} \right] - c_A^2Tr \left[ \gamma^\mu\gamma^5\slashed{k}'\gamma^{5}\gamma^{\nu}\slashed{k} \right] - 2c_vc_ATr\left[ \gamma^\mu\gamma^5\slashed{k}'\gamma^\nu\slashed{k} \right]\\
		=& c_v^2Tr\left[ \gamma^\mu\slashed{k}'\gamma^{\nu}\slashed{k} \right] - c_A^2Tr \left[- \gamma^\mu\slashed{k}'\gamma^5\gamma^{5}\gamma^{\nu}\slashed{k} \right]-2c_vc_ATr\left[ \gamma^\mu\gamma^5\slashed{k}'\gamma^\nu\slashed{k} \right] \\
		=& c_v^2Tr\left[ \gamma^\mu\slashed{k}'\gamma^{\nu}\slashed{k} \right] + c_A^2Tr \left[ \gamma^\mu\slashed{k}'\gamma^{\nu}\slashed{k} \right] -2c_vc_ATr\left[ \gamma^\mu\gamma^5\slashed{k}'\gamma^\nu\slashed{k} \right]\\
		=& Tr \left[ \gamma^\mu\slashed{k}'\gamma^{\nu}\slashed{k} \right](c_v^2+c_A^2)-2c_vc_ATr\left[ \gamma^\mu\gamma^5\slashed{k}'\gamma^\nu\slashed{k} \right]
		\end{split}
	\]
	From the trace theorems in Chapter 6 of the text, we see that our last trace is proportional to the levi-cevita tensor which is antisymmetric. Both $g_{\mu\nu}$ and $q_\mu q_\nu$ are symmetric. The inner product between an antisymmetric and a symmetric tensor is zero.
	\\
	\\
	Since we are working the rest frame of the boson, we have the following four-momenta:
	\[
		P_X^\sigma = (M_X,0),\; P_1^\sigma=(\frac{M_X}{2},\vec{p}),\; P_2^\sigma = (\frac{M_X}{2},-\vec{p})
	\]
	Putting all this together:
	\[
		\begin{split} 
		\overline{\abs{\mathcal{M}}^2} =& \frac{g_X^2}{12} \left( -g_{\mu\nu} +q_\mu q_\nu \right)Tr \left[ \gamma^\mu\slashed{k}'\gamma^{\nu}\slashed{k} \right](c_v^2+c_A^2) \\
		=&\frac{g_X^2}{12} \left( -g_{\mu\nu}Tr \left[ \gamma^\mu\slashed{k}'\gamma^{\nu}\slashed{k} \right] +q_\mu q_\nu Tr \left[ \gamma^\mu\slashed{k}'\gamma^{\nu}\slashed{k} \right] \right)(c_v^2+c_A^2) \\
		=& \frac{g_X^2}{12} \left( -g_{\mu\nu}4(k'^\mu k^\nu-g^{\mu\nu}(k'\cdot k)+k'^\mu k^\nu) +q_\mu q_\nu 4(k'^\mu k^\nu-g^{\mu\nu}(k'\cdot k)+k'^\mu k^\nu) \right)(c_v^2+c_A^2) \\
		=& -\frac{g_X^2}{12} 4\left( k'_\nu k^\nu-4(k'\cdot k)+k'_\nu k^\nu +\left(\frac{M_X^2}{2}\right)^2+\left(\frac{M_X^2}{2}\right)^2-2\frac{M_X^2}{4}M_X^2 \right)(c_v^2+c_A^2) \\
		=& -\frac{g_X^2}{3} \left( 2(k'\cdot k)-4(k'\cdot k)\right)(c_v^2+c_A^2) \\
		=& \frac{g_X^2}{3} \left( 2(k'\cdot k)\right)(c_v^2+c_A^2) 
		\end{split} 
	\]
	Now we calculate the decay rate, $\Gamma$, but first let's go ahead and evaluate that trace
	\[
		\begin{split}
		\Gamma(X\rightarrow f_1\bar{f}_2) =& \frac{p_f}{32\pi^2M_X^2}\int \overline{\abs{\mathcal{M}}^2}d\Omega \\
		=&\frac{M_X}{2}\frac{2g_X^2}{96\pi^2M_X^2}(c_v^2+c_A^2)\int k'\cdot k d\Omega \\
		=& \frac{g_X^2}{96\pi^2M_X}(c_v^2+c_A^2)4\pi \frac{M_X^2}{2} \\
		=& \frac{g_X^2}{48\pi}(c_v^2+c_A^2) M_X
		\end{split}
	\]
\end{homeworkProblem}

\pagebreak

\begin{homeworkProblem}
	Using the result of the previous problem, compute the total widths and branching ratios for the $Z$ and $W$ decays into all possible final-state fermions. Use $\sin^2\theta_W=0.23$, $M_Z=91\;GeV$, and $G_F=1.17\times 10^{-5}\;GeV^{-2}$.
	\\


	\textbf{Solution}
	\\
	\\
	First, the $Z$ decays. The $Z$ boson does not change flavor, so all decays must be of the same flavor, have zero net charge, and conserve lepton number, meaning for example that you can't get a final product containing two neutrinos that are not antiparticle versions of each other. Note that the top quark is too heavy to be a decay product from the $Z$ boson.
	\begin{center}
		\begin{tabular}{|c|c|c|c|c|c|} 
			quarks & $Z\rightarrow u\bar{u}$ & $Z\rightarrow d\bar{d}$ & $Z\rightarrow c\bar{c}$ & $Z\rightarrow s\bar{s}$& $Z\rightarrow b\bar{b}$ \\  
			leptons & $Z\rightarrow e^+e^-$ & $Z\rightarrow \mu^+\mu^-$ & $Z\rightarrow \tau^+\tau^-$&X&X \\
			neutrinos & $Z\rightarrow \nu_e\bar{\nu}_e$ & $Z\rightarrow \nu_\mu\bar{\nu}_\mu$ & $Z\rightarrow \nu_\tau\bar{\nu}_\tau$ & X&X    
		\end{tabular}
	\end{center}

	The relation between $G_F$ and $g$ is given by \( G_F=\frac{\sqrt{2}}{8}\frac{g^2}{M_X^2} \). So in the decay equation, we write:
	\[
		\Gamma(X\rightarrow f_1\bar{f}_2)=\frac{8G_FM_X^2}{\sqrt{2}48\pi}(c_v^2+c_A^2)M_X=\frac{G_FM_X^3}{6\pi\sqrt{2}}(c_v^2+c_A^2)
	\]
	The decay widths: (all of these have the form $Z\rightarrow ff$, so in the equations I just put which type of fermion it decays to). The values of the constant, $c_v$ and $c_A$ come from the textbook, p. 301 using $\sin^2\theta_W=0.23$.
	\[
		\begin{split}
		\Gamma(q^+) =& \frac{G_FM_Z^3}{6\pi\sqrt{2}}(c_{v,q^+}^2+c_{A,q^+}^2) = 0.0950 \\
		\Gamma(q^-) =& \frac{G_FM_Z^3}{6\pi\sqrt{2}}(c_{v,q^-}^2+c_{A,q^-}^2) = 0.123 \\
		\Gamma(l) =& \frac{G_FM_Z^3}{6\pi\sqrt{2}}(c_{v,l}^2+c_{A,l}^2) = 0.0832 \\
		\Gamma(\nu) =& \frac{G_FM_Z^3}{6\pi\sqrt{2}}(c_{v,\nu}^2+c_{A,\nu}^2) = 0.165 \\
		\Gamma_Z =& 0.467
		\end{split} 
	\]
	Then the branching ratios are:
	\[
		\begin{split} 
		\Gamma(q^+)/\Gamma_Z =& 0.20\\
		\Gamma(q^-)/\Gamma_Z =& 0.26\\
		\Gamma(l)/\Gamma_Z =& 0.18\\
		\Gamma(\nu)/\Gamma_Z =& 0.35
		\end{split}
	\]
	Now for the $W$ decays. We have to be careful to separate the positive from the negative $W$ boson. We also have to consider that the $W$ boson allows for flavor changing interactions.
	\begin{center}
		\begin{tabular}{|c|c|c|c|c|c|c|} 
			quarks & $W^+\rightarrow u\bar{d}$ & $W^+\rightarrow u\bar{s}$ & $W^+\rightarrow u\bar{b}$ & $W^+\rightarrow c\bar{d}$ & $W^+\rightarrow c\bar{s}$ & $W^+\rightarrow c\bar{b}$ \\  
			quarks & $W^-\rightarrow \bar{u}d$ & $W^-\rightarrow \bar{u}s$ & $W^-\rightarrow \bar{u}b$ & $W^-\rightarrow \bar{c}d$ & $W^+\rightarrow \bar{c}s$ & $W^+\rightarrow \bar{c}b$ \\ 
			lepton $\nu$ & $W^+\rightarrow e^+\nu_e$ & $W^+\rightarrow \mu^+\nu_\mu$ & $W^+\rightarrow \tau^+\nu_\tau$ & X & X & X \\
			lepton $\nu$ & $W^-\rightarrow e^-\bar{\nu}_e$ & $W^-\rightarrow \mu^-\bar{\nu}_\mu$ & $W^-\rightarrow \tau^-\bar{\nu}_\tau$ & X & X & X \\		    
		\end{tabular}
	\end{center}
	Since $W$ goes to particles and antiparticles the same way, just with everything conjugated, the decay rates will be the same. So only one from each pair needs to be calculated. In the case of the quarks, we have to include the strength of the flavor-changing interactions from the CKM matrix. Because these are different for different quark pairs, unlike with the $Z$ decay, we have to calculate different decay widths for each quark pair. Also note, $c_A=c_v=1/\sqrt{2}$ for $W$ decays.
	\[
		\begin{split} 
		\Gamma(l^+\nu)=&\frac{G_FM_W^3}{12\pi\sqrt{2}} = 0.114\\
		\Gamma(u\bar{d})=&V_{ud}^2\frac{G_FM_W^3}{12\pi\sqrt{2}} = 0.108\\
		\Gamma(u\bar{s})=&V_{us}^2\frac{G_FM_W^3}{12\pi\sqrt{2}} = 0.00577\\
		\Gamma(u\bar{b})=&V_{ub}^2\frac{G_FM_W^3}{12\pi\sqrt{2}} = 1.41\times 10^{-6}\\
		\Gamma(c\bar{d})=&V_{cd}^2\frac{G_FM_W^3}{12\pi\sqrt{2}} = 0.00577\\
		\Gamma(c\bar{s})=&V_{cs}^2\frac{G_FM_W^3}{12\pi\sqrt{2}} = 0.108\\
		\Gamma(c\bar{b})=&V_{cb}^2\frac{G_FM_W^3}{12\pi\sqrt{2}} = 0.000194\\
		\Gamma_W =& 0.684
		\end{split} 
	\]
	The branching ratios are then:
	\[
		\begin{split} 
		\Gamma(l^+\nu)/\Gamma_W =& 0.167 \\
		\Gamma(u\bar{d})/\Gamma_W =& 0.158 \\
		\Gamma(u\bar{b})/\Gamma_W =& 0.00844 \\
		\Gamma(u\bar{s}/\Gamma_W =& 2.05\times10^{-6} \\
		\Gamma(c\bar{d})/\Gamma_W =& 0.00844 \\
		\Gamma(c\bar{s})/\Gamma_W =& 0.158 \\
		\Gamma(c\bar{b})/\Gamma_W =& 0.000283 
		\end{split} 
	\]
\end{homeworkProblem}

\pagebreak

\begin{homeworkProblem}
	The Lagrangian for three interacting real fields $\phi_1$, $\phi_2$, $\phi_3$ is
	\[
		\mathcal{L}=\frac{1}{2}(\partial_\mu\phi_i)^2-\frac{1}{2}\mu^2\phi_i^2-\frac{1}{4}\lambda(\phi_i^2)^2
	\]
	with $\mu^2<0$ and $\lambda>0$, and where summation of $\phi_i^2$ over $i$ is implied. Show that it describes a massive field of mass $\sqrt{-2\mu^2}$ and two massless Goldstone bosons.
	\\
	\\
	\textbf{Solution}
	\\
	\\
\end{homeworkProblem}

\end{document}

