\documentclass{article}

\usepackage{hyperref}
\usepackage{fancyhdr}
\usepackage{braket}
\usepackage{extramarks}
\usepackage{amsmath}
\usepackage{slashed}
\usepackage{amsthm}
\usepackage{amsfonts}
\usepackage{tikz}
\usepackage[plain]{algorithm}
\usepackage{algpseudocode}
\usepackage{mathtools}
\usepackage{graphicx}
\graphicspath{ {./Images/} }

\DeclarePairedDelimiter\abs{\lvert}{\rvert}%
\DeclarePairedDelimiter\norm{\lVert}{\rVert}%

\makeatletter
\let\oldabs\abs
\def\abs{\@ifstar{\oldabs}{\oldabs*}}
%
\let\oldnorm\norm
\def\norm{\@ifstar{\oldnorm}{\oldnorm*}}
\makeatother

\newcommand*{\Value}{\frac{1}{2}x^2}%

\usetikzlibrary{automata,positioning}

%
% Basic Document Settings
%

\topmargin=-0.45in
\evensidemargin=0in
\oddsidemargin=0in
\textwidth=6.5in
\textheight=9.0in
\headsep=0.25in

\linespread{1.1}

\pagestyle{fancy}
\lhead{\hmwkAuthorName}
\chead{\hmwkClass\ (\hmwkClassInstructor\ \hmwkClassTime): \hmwkTitle}
\rhead{\firstxmark}
\lfoot{\lastxmark}
\cfoot{\thepage}

\renewcommand\headrulewidth{0.4pt}
\renewcommand\footrulewidth{0.4pt}

\setlength\parindent{0pt}

%
% Create Problem Sections
%

\newcommand{\enterProblemHeader}[1]{
    \nobreak\extramarks{}{Problem \arabic{#1} continued on next page\ldots}\nobreak{}
    \nobreak\extramarks{Problem \arabic{#1} (continued)}{Problem \arabic{#1} continued on next page\ldots}\nobreak{}
}

\newcommand{\exitProblemHeader}[1]{
    \nobreak\extramarks{Problem \arabic{#1} (continued)}{Problem \arabic{#1} continued on next page\ldots}\nobreak{}
    \stepcounter{#1}
    \nobreak\extramarks{Problem \arabic{#1}}{}\nobreak{}
}

\setcounter{secnumdepth}{0}
\newcounter{partCounter}
\newcounter{homeworkProblemCounter}
\setcounter{homeworkProblemCounter}{1}
\nobreak\extramarks{Problem \arabic{homeworkProblemCounter}}{}\nobreak{}

%
% Homework Problem Environment
%
% This environment takes an optional argument. When given, it will adjust the
% problem counter. This is useful for when the problems given for your
% assignment aren't sequential. See the last 3 problems of this template for an
% example.
%
\newenvironment{homeworkProblem}[1][-1]{
    \ifnum#1>0
        \setcounter{homeworkProblemCounter}{#1}
    \fi
    \section{Problem \arabic{homeworkProblemCounter}}
    \setcounter{partCounter}{1}
    \enterProblemHeader{homeworkProblemCounter}
}{
    \exitProblemHeader{homeworkProblemCounter}
}

%
% Homework Details
%   - Title
%   - Due date
%   - Class
%   - Section/Time
%   - Instructor
%   - Author
%

\newcommand{\hmwkTitle}{Homework\ \#12}
\newcommand{\hmwkDueDate}{April 21, 2020}
\newcommand{\hmwkClass}{Physics 926}
\newcommand{\hmwkClassTime}{}
\newcommand{\hmwkClassInstructor}{Professor Ken Bloom}
\newcommand{\hmwkAuthorName}{\textbf{Robert Tabb}}

%
% Title Page
%

\title{
    \vspace{2in}
    \textmd{\textbf{\hmwkClass:\ \hmwkTitle}}\\
    \normalsize\vspace{0.1in}\small{Due\ on\ \hmwkDueDate\ at 5pm}\\
    \vspace{0.1in}\large{\textit{\hmwkClassInstructor\ \hmwkClassTime}}
    \vspace{3in}
}

\author{\hmwkAuthorName}
\date{}

\renewcommand{\part}[1]{\textbf{\large Part \Alph{partCounter}}\stepcounter{partCounter}\\}

%
% Various Helper Commands
%

% Useful for algorithms
\newcommand{\alg}[1]{\textsc{\bfseries \footnotesize #1}}

% For derivatives
\newcommand{\deriv}[1]{\frac{\mathrm{d}}{\mathrm{d}x} (#1)}

% For partial derivatives
\newcommand{\pderiv}[2]{\frac{\partial}{\partial #1} (#2)}

% Integral dx
\newcommand{\dx}{\mathrm{d}x}

% Alias for the Solution section header
\newcommand{\solution}{\textbf{\large Solution}}

% Probability commands: Expectation, Variance, Covariance, Bias
\newcommand{\E}{\mathrm{E}}
\newcommand{\Var}{\mathrm{Var}}
\newcommand{\Cov}{\mathrm{Cov}}
\newcommand{\Bias}{\mathrm{Bias}}

\begin{document}

\maketitle

\pagebreak

\begin{homeworkProblem}
	If the vertex factor for the decay of a vector boson $X$ into two spin-1/2 fermions $f_1$ and $f_2$ is
	\[
		-igx\gamma^\mu\frac{1}{2}(c_v-c_A\gamma^5)
	\]
	then show that
	\[
		\Gamma(X\rightarrow f_1\bar{f}_2)=\frac{g_X^2}{48\pi}(c_v^2+c_A^2)M_X
	\]
	where $M_X$ is the mass of the boson and where we have neglected the mass of the fermions. Hints: use
	\[
		\sum_{\lambda} \epsilon_\mu^{(\lambda)*} \epsilon_\nu^\lambda=-g_{\mu\nu}+\frac{p_\mu p_\nu}{M^2}
	\]
	to show that after summing over the fermions and averaging over the boson spins ,
	\[
		\overline{\abs{\mathcal{M}}^2}=\frac{1}{12} g_X^2(c_v^2+c_A^2)(-g_{\mu\nu})Tr(\gamma^\mu\slashed{k}\gamma^\nu\slashed{k}')
	\]
	where $k$ and $k'$ are the four-momenta of the fermions. Work in the boson rest frame, and use
	\[
		\Gamma(X\rightarrow f_1\bar{f}_2)\frac{p_f}{32\pi^2m_X^2}\int \overline{\abs{\mathcal{M}}^2} d\Omega
	\]
\textbf{Solution}
\\
\\
	Using the vertex factor given in the problem, begin by writing the matrix element
	\[
		\begin{split}
		\mathcal{M}=&\bar{u}(k)\left[-ig_X\gamma^\mu\frac{1}{2}(c_v-c_A\gamma^5)\right]v(k')\epsilon_\mu \\
		=&-\frac{ig_X}{2}\left[\bar{u}(k)\gamma^\mu(c_v-c_A\gamma^5)v(k')\epsilon_\mu\right] \\
		\abs{\mathcal{M}}^2 =&\frac{g_X^2}{4}\left[\bar{u}(k)\gamma^\mu(c_v-c_A\gamma^5)v(k')\epsilon_\mu\right]\left[\bar{u}(k)\gamma^\nu(c_v-c_A\gamma^5)v(k')\epsilon_\nu\right]^*
		\end{split}
	\]
	We now need to write the average by summing over the spins and polarizations. In this step, assume the fermion masses can be neglected and use the formula: \( \sum_s [\bar{u}(a)\Gamma_1v(b)][\bar{u}(a)\Gamma_1v(b)]^* = Tr[\Gamma_1\slashed{b}\bar{\Gamma}_2\slashed{a}]  \), recalling that \( \bar{\Gamma} =\gamma^0\Gamma^\dagger\gamma^0 \)
	\[
		\begin{split}
		\overline{\abs{\mathcal{M}}^2} =& \frac{1}{3}\sum_{s,\lambda}\abs{\mathcal{M}}^2 \\
		=&\frac{g_X^2}{12}\sum_\lambda\left( \epsilon_\mu \epsilon_\nu^* \right) \sum_s \left[\bar{u}(k)\gamma^\mu(c_v-c_A\gamma^5)v(k')\right]\left[\bar{u}(k)\gamma^\nu(c_v-c_A\gamma^5)v(k')\right]^* \\
		=&\frac{g_X^2}{12} \left( -g_{\mu\nu}+\frac{q_\mu q_\nu}{M_X^2} \right)Tr\left[ \gamma^\mu(c_v-c_A\gamma^5)\slashed{k}'\gamma^0(\gamma^\nu(c_v-c_A\gamma^5))^\dagger\gamma^0\slashed{k} \right] \\
		=& \frac{g_X^2}{12} \left( -g_{\mu\nu}+\frac{q_\mu q_\nu}{M_X^2} \right)Tr\left[ (c_v\gamma^\mu\slashed{k}'-c_A\gamma^\mu\gamma^5\slashed{k}') (c_v\gamma^0\gamma^{\nu\dagger}\gamma^0\slashed{k}-c_A\gamma^0\gamma^{5\dagger}\gamma^{\nu\dagger}\gamma^0\slashed{k}) \right] \\
		=& \frac{g_X^2}{12} \left( -g_{\mu\nu}+\frac{q_\mu q_\nu}{M_X^2} \right) Tr\left[ (c_v\gamma^\mu\slashed{k}'-c_A\gamma^\mu\gamma^5\slashed{k}')    (c_v\gamma^{\nu}\slashed{k}+c_A\gamma^{5}\gamma^{\nu}\slashed{k}) \right] \\
		\end{split}
	\]
	Where the plus sign in the last line comes from letting \(\gamma^0\gamma^5\rightarrow \gamma^5\gamma^0  \)
	Now let's evaluate that trace
	\[
		\begin{split} 
		Tr\left[(c_v\gamma^\mu\slashed{k}'-c_A\gamma^\mu\gamma^5\slashed{k}')    (c_v\gamma^{\nu}\slashed{k}+c_A\gamma^{5}\gamma^{\nu}\slashed{k})  \right] =& Tr\left[ c_v^2\gamma^\mu\slashed{k}'\gamma^{\nu}\slashed{k}-c_A^2\gamma^\mu\gamma^5\slashed{k}'\gamma^{5}\gamma^{\nu}\slashed{k}  \right] \\
		=& c_v^2Tr\left[ \gamma^\mu\slashed{k}'\gamma^{\nu}\slashed{k} \right] - c_A^2Tr \left[ \gamma^\mu\gamma^5\slashed{k}'\gamma^{5}\gamma^{\nu}\slashed{k} \right] \\
		=& c_v^2Tr\left[ \gamma^\mu\slashed{k}'\gamma^{\nu}\slashed{k} \right] - c_A^2Tr \left[- \gamma^\mu\slashed{k}'\gamma^5\gamma^{5}\gamma^{\nu}\slashed{k} \right] \\
		=& c_v^2Tr\left[ \gamma^\mu\slashed{k}'\gamma^{\nu}\slashed{k} \right] + c_A^2Tr \left[ \gamma^\mu\slashed{k}'\gamma^{\nu}\slashed{k} \right] \\
		=& Tr \left[ \gamma^\mu\slashed{k}'\gamma^{\nu}\slashed{k} \right](c_v^2+c_A^2)
		\end{split}
	\]
	Putting all this together:
	\[
		\overline{\abs{\mathcal{M}}^2} = \frac{g_X^2}{12} \left( -g_{\mu\nu} \right)Tr \left[ \gamma^\mu\slashed{k}'\gamma^{\nu}\slashed{k} \right](c_v^2+c_A^2)
	\]
	Now we calculate the decay rate, $\Gamma$, but first let's go ahead and evaluate that trace
	\[
		\begin{split}
		\Gamma(X\rightarrow f_1\bar{f}_2) = \frac{p_f}{32\pi^2M_X^2}\int \overline{\abs{\mathcal{M}}^2}d\Omega \\
		\end{split}
	\]
\end{homeworkProblem}

\pagebreak

\begin{homeworkProblem}
	Using the result of the previous problem, compute the total widths and branching ratios for the $Z$ and $W$ decays into all possible final-state fermions. Use $\sin^2\theta_W=0.23$, $M_Z=91\;GeV$, and $G_F=1.17\times 10^{-5}\;GeV^{-2}$.
	\\


	\textbf{Solution}
	\\
	\\
	First, the $Z$ decays. The $Z$ boson does not change flavor, so all decays must be of the same flavor, have zero net charge, and conserve lepton number, meaning for example that you can't get a final product containing two neutrinos that are not antiparticle versions of each other. Note that the top quark is too heavy to be a decay product from the $Z$ boson.
	\begin{center}
		\begin{tabular}{|c|c|c|c|c|c|} 
			quarks & $Z\rightarrow u\bar{u}$ & $Z\rightarrow d\bar{d}$ & $Z\rightarrow c\bar{c}$ & $Z\rightarrow s\bar{s}$& $Z\rightarrow b\bar{b}$ \\  
			leptons & $Z\rightarrow e^+e^-$ & $Z\rightarrow \mu^+\mu^-$ & $Z\rightarrow \tau^+\tau^-$&X&X \\
			neutrinos & $Z\rightarrow \nu_e\bar{\nu}_e$ & $Z\rightarrow \nu_\mu\bar{\nu}_\mu$ & $Z\rightarrow \nu_\tau\bar{\nu}_\tau$ & X&X    
		\end{tabular}
	\end{center}

	The decay widths: (all of these have the form $Z\rightarrow ff$, so in the equations I just put which type of fermion it decays to)
	\[
		\begin{split}
		\Gamma(q^+) =& \frac{G^2M_Z(c_{v,q^+}^2+c_{A,q^+}^2)}{48\pi} \\
		=& 2.37\times 10^{-11} \\
		\Gamma(q^-) =& \frac{G^2M_Z(c_{v,q^-}^2+c_{A,q^-}^2)}{48\pi} \\
		=& 3.08\times 10^{-11} \\
		\Gamma(l) =& \frac{G^2M_Z(c_{v,l}^2+c_{A,l}^2)}{48\pi} \\
		=& 2.08\times 10^{-11} \\
		\Gamma(\nu) =& \frac{G^2M_Z(c_{v,\nu}^2+c_{A,\nu}^2)}{48\pi} \\
		=& 4.13\times 10^{-11} \\
		\Gamma_Z =& 2\Gamma(q^+)+3(\Gamma(q^-)+\Gamma(l)+\Gamma(\nu))	\\
		=& 3.24\times10^{-10}
		\end{split} 
	\]
	Then the branching ratios are:
	\[
		\begin{split} 
		\Gamma(q^+)/\Gamma_Z =& 0.073\\
		\Gamma(q^-)/\Gamma_Z =& 0.094\\
		\Gamma(l)/\Gamma_Z =& 0.064\\
		\Gamma(\nu)/\Gamma_Z =& 0.13
		\end{split}
	\]
\end{homeworkProblem}

\pagebreak

\begin{homeworkProblem}
\textbf{Solution}
\end{homeworkProblem}

\end{document}

