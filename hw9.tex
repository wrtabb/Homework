\documentclass{article}

\usepackage{fancyhdr}
\usepackage{extramarks}
\usepackage{amsmath}
\usepackage{amsthm}
\usepackage{amsfonts}
\usepackage{tikz}
\usepackage[plain]{algorithm}
\usepackage{algpseudocode}
\usepackage{mathtools}

\DeclarePairedDelimiter\abs{\lvert}{\rvert}%
\DeclarePairedDelimiter\norm{\lVert}{\rVert}%

\makeatletter
\let\oldabs\abs
\def\abs{\@ifstar{\oldabs}{\oldabs*}}
%
\let\oldnorm\norm
\def\norm{\@ifstar{\oldnorm}{\oldnorm*}}
\makeatother

\newcommand*{\Value}{\frac{1}{2}x^2}%

\usetikzlibrary{automata,positioning}

%
% Basic Document Settings
%

\topmargin=-0.45in
\evensidemargin=0in
\oddsidemargin=0in
\textwidth=6.5in
\textheight=9.0in
\headsep=0.25in

\linespread{1.1}

\pagestyle{fancy}
\lhead{\hmwkAuthorName}
\chead{\hmwkClass\ (\hmwkClassInstructor\ \hmwkClassTime): \hmwkTitle}
\rhead{\firstxmark}
\lfoot{\lastxmark}
\cfoot{\thepage}

\renewcommand\headrulewidth{0.4pt}
\renewcommand\footrulewidth{0.4pt}

\setlength\parindent{0pt}

%
% Create Problem Sections
%

\newcommand{\enterProblemHeader}[1]{
    \nobreak\extramarks{}{Problem \arabic{#1} continued on next page\ldots}\nobreak{}
    \nobreak\extramarks{Problem \arabic{#1} (continued)}{Problem \arabic{#1} continued on next page\ldots}\nobreak{}
}

\newcommand{\exitProblemHeader}[1]{
    \nobreak\extramarks{Problem \arabic{#1} (continued)}{Problem \arabic{#1} continued on next page\ldots}\nobreak{}
    \stepcounter{#1}
    \nobreak\extramarks{Problem \arabic{#1}}{}\nobreak{}
}

\setcounter{secnumdepth}{0}
\newcounter{partCounter}
\newcounter{homeworkProblemCounter}
\setcounter{homeworkProblemCounter}{1}
\nobreak\extramarks{Problem \arabic{homeworkProblemCounter}}{}\nobreak{}

%
% Homework Problem Environment
%
% This environment takes an optional argument. When given, it will adjust the
% problem counter. This is useful for when the problems given for your
% assignment aren't sequential. See the last 3 problems of this template for an
% example.
%
\newenvironment{homeworkProblem}[1][-1]{
    \ifnum#1>0
        \setcounter{homeworkProblemCounter}{#1}
    \fi
    \section{Problem \arabic{homeworkProblemCounter}}
    \setcounter{partCounter}{1}
    \enterProblemHeader{homeworkProblemCounter}
}{
    \exitProblemHeader{homeworkProblemCounter}
}

%
% Homework Details
%   - Title
%   - Due date
%   - Class
%   - Section/Time
%   - Instructor
%   - Author
%

\newcommand{\hmwkTitle}{Homework\ \#9}
\newcommand{\hmwkDueDate}{March 31, 2020}
\newcommand{\hmwkClass}{Physics 926}
\newcommand{\hmwkClassTime}{}
\newcommand{\hmwkClassInstructor}{Professor Ken Bloom}
\newcommand{\hmwkAuthorName}{\textbf{Robert Tabb}}

%
% Title Page
%

\title{
    \vspace{2in}
    \textmd{\textbf{\hmwkClass:\ \hmwkTitle}}\\
    \normalsize\vspace{0.1in}\small{Due\ on\ \hmwkDueDate\ at 5pm}\\
    \vspace{0.1in}\large{\textit{\hmwkClassInstructor\ \hmwkClassTime}}
    \vspace{3in}
}

\author{\hmwkAuthorName}
\date{}

\renewcommand{\part}[1]{\textbf{\large Part \Alph{partCounter}}\stepcounter{partCounter}\\}

%
% Various Helper Commands
%

% Useful for algorithms
\newcommand{\alg}[1]{\textsc{\bfseries \footnotesize #1}}

% For derivatives
\newcommand{\deriv}[1]{\frac{\mathrm{d}}{\mathrm{d}x} (#1)}

% For partial derivatives
\newcommand{\pderiv}[2]{\frac{\partial}{\partial #1} (#2)}

% Integral dx
\newcommand{\dx}{\mathrm{d}x}

% Alias for the Solution section header
\newcommand{\solution}{\textbf{\large Solution}}

% Probability commands: Expectation, Variance, Covariance, Bias
\newcommand{\E}{\mathrm{E}}
\newcommand{\Var}{\mathrm{Var}}
\newcommand{\Cov}{\mathrm{Cov}}
\newcommand{\Bias}{\mathrm{Bias}}

\begin{document}

\maketitle

\pagebreak

\begin{homeworkProblem}
    Show that in $\pi$$\rightarrow$$\mu$$\nu$ decay, \(\abs{\vec{p}_\mu} = \abs{\vec{p}_\nu} = (m_{\pi}^{2} - m_{\mu}^{2}) / 2m_{\pi}\)
\\
\\
    \textbf{Solution}

    Start by defining the momentum four-vectors for each particle in the center of momentum frame. Note that $\sigma$ is the index while $\mu$, $\nu$, and $\pi$ are the names of the particles.
    \\
	\[
		\begin{split}
		P_\pi^\sigma=(m_\pi, \vec{0})
		\\
		P_\mu^\sigma=(E_\mu, \vec{p})
		\\
		P_\nu^\sigma=(E_\nu,\vec{-p})
		\end{split}
	\]

	Due to conservation of four-momentum, we can write:
	\[
		\begin{split}
		P_\pi^\sigma = P_\mu^\sigma + P_\nu^\sigma
		\\
		P_\pi^\sigma - P_\nu^\sigma = P_\mu^\sigma
		\end{split}
	\]
	
	Now we contract each side with itself which we can do since this operation is Lorentz invariant:
	\[
		\begin{split}
		(P_\pi - P_\nu)^\sigma (P_\pi - P_\nu)_\sigma = P_\mu^\sigma P_{\mu,\sigma}
		\\
		P_\pi^2 + P_\nu^2 - 2P_\nu^\sigma P_{\pi,\sigma} = m_\mu^2
		\end{split}
	\]
	Since we are assuming the the neutrino is massless, \(P_\nu^2 = 0\) and \(E_\nu = \abs{\vec{p}}\)
	
	\[
		\begin{split}
		m_\pi^2-2\abs{\vec{p}} m_\pi = m_\mu^2
		\\
		-2\abs{\vec{p}} m_\pi = m_\mu^2 - m_\pi^2
		\\
		\abs{\vec{p}} = \frac{m_\pi^2 - m_\mu^2}{2m_\pi}
		\end{split}
	\]
\end{homeworkProblem}

\pagebreak

\begin{homeworkProblem}
   (H\&M exercise 12.13) Predict the ratio of the K $\rightarrow$ e $\bar{\nu}_e$ and K $\rightarrow$ $\mu$ $\bar{\nu}_\mu$ decay rates. Given that the lifetime of the K is \(\tau = 1.2 \times 10^8 s\) and the K $\rightarrow \mu \nu$ branching ratio is 64\%, estimate the decay constant f$_K$. Comment on your assumptions and your result.
   \\
   \\
   \textbf{Solution}
   
\end{homeworkProblem}

\pagebreak

\begin{homeworkProblem}
   
\end{homeworkProblem}

\pagebreak

\begin{homeworkProblem}
    
\end{homeworkProblem}

\pagebreak

\begin{homeworkProblem}
    
\end{homeworkProblem}

\pagebreak

\begin{homeworkProblem}
	\textbf{Part a}
	\\
	\textbf{Part b}
\end{homeworkProblem}


\end{document}
