\documentclass{article}

\usepackage{fancyhdr}
\usepackage{braket}
\usepackage{extramarks}
\usepackage{amsmath}
\usepackage{amsthm}
\usepackage{amsfonts}
\usepackage{tikz}
\usepackage[plain]{algorithm}
\usepackage{algpseudocode}
\usepackage{mathtools}
\usepackage{graphicx}
\graphicspath{ {./Images/} }

\DeclarePairedDelimiter\abs{\lvert}{\rvert}%
\DeclarePairedDelimiter\norm{\lVert}{\rVert}%

\makeatletter
\let\oldabs\abs
\def\abs{\@ifstar{\oldabs}{\oldabs*}}
%
\let\oldnorm\norm
\def\norm{\@ifstar{\oldnorm}{\oldnorm*}}
\makeatother

\newcommand*{\Value}{\frac{1}{2}x^2}%

\usetikzlibrary{automata,positioning}

%
% Basic Document Settings
%

\topmargin=-0.45in
\evensidemargin=0in
\oddsidemargin=0in
\textwidth=6.5in
\textheight=9.0in
\headsep=0.25in

\linespread{1.1}

\pagestyle{fancy}
\lhead{\hmwkAuthorName}
\chead{\hmwkClass\ (\hmwkClassInstructor\ \hmwkClassTime): \hmwkTitle}
\rhead{\firstxmark}
\lfoot{\lastxmark}
\cfoot{\thepage}

\renewcommand\headrulewidth{0.4pt}
\renewcommand\footrulewidth{0.4pt}

\setlength\parindent{0pt}

%
% Create Problem Sections
%

\newcommand{\enterProblemHeader}[1]{
    \nobreak\extramarks{}{Problem \arabic{#1} continued on next page\ldots}\nobreak{}
    \nobreak\extramarks{Problem \arabic{#1} (continued)}{Problem \arabic{#1} continued on next page\ldots}\nobreak{}
}

\newcommand{\exitProblemHeader}[1]{
    \nobreak\extramarks{Problem \arabic{#1} (continued)}{Problem \arabic{#1} continued on next page\ldots}\nobreak{}
    \stepcounter{#1}
    \nobreak\extramarks{Problem \arabic{#1}}{}\nobreak{}
}

\setcounter{secnumdepth}{0}
\newcounter{partCounter}
\newcounter{homeworkProblemCounter}
\setcounter{homeworkProblemCounter}{1}
\nobreak\extramarks{Problem \arabic{homeworkProblemCounter}}{}\nobreak{}

%
% Homework Problem Environment
%
% This environment takes an optional argument. When given, it will adjust the
% problem counter. This is useful for when the problems given for your
% assignment aren't sequential. See the last 3 problems of this template for an
% example.
%
\newenvironment{homeworkProblem}[1][-1]{
    \ifnum#1>0
        \setcounter{homeworkProblemCounter}{#1}
    \fi
    \section{Problem \arabic{homeworkProblemCounter}}
    \setcounter{partCounter}{1}
    \enterProblemHeader{homeworkProblemCounter}
}{
    \exitProblemHeader{homeworkProblemCounter}
}

%
% Homework Details
%   - Title
%   - Due date
%   - Class
%   - Section/Time
%   - Instructor
%   - Author
%

\newcommand{\hmwkTitle}{Homework\ \#4}
\newcommand{\hmwkDueDate}{April 13, 2020}
\newcommand{\hmwkClass}{Physics 916}
\newcommand{\hmwkClassTime}{}
\newcommand{\hmwkClassInstructor}{Professor Jean Marcel Ngoko}
\newcommand{\hmwkAuthorName}{\textbf{Robert Tabb}}

%
% Title Page
%

\title{
    \vspace{2in}
    \textmd{\textbf{\hmwkClass:\ \hmwkTitle}}\\
    \normalsize\vspace{0.1in}\small{Due\ on\ \hmwkDueDate\ at 5pm}\\
    \vspace{0.1in}\large{\textit{\hmwkClassInstructor\ \hmwkClassTime}}
    \vspace{3in}
}

\author{\hmwkAuthorName}
\date{}

\renewcommand{\part}[1]{\textbf{\large Part \Alph{partCounter}}\stepcounter{partCounter}\\}

%
% Various Helper Commands
%

% Useful for algorithms
\newcommand{\alg}[1]{\textsc{\bfseries \footnotesize #1}}

% For derivatives
\newcommand{\deriv}[1]{\frac{\mathrm{d}}{\mathrm{d}x} (#1)}

% For partial derivatives
\newcommand{\pderiv}[2]{\frac{\partial}{\partial #1} (#2)}

% Integral dx
\newcommand{\dx}{\mathrm{d}x}

% Alias for the Solution section header
\newcommand{\solution}{\textbf{\large Solution}}

% Probability commands: Expectation, Variance, Covariance, Bias
\newcommand{\E}{\mathrm{E}}
\newcommand{\Var}{\mathrm{Var}}
\newcommand{\Cov}{\mathrm{Cov}}
\newcommand{\Bias}{\mathrm{Bias}}

\begin{document}

\maketitle

\pagebreak

\begin{homeworkProblem}
	Consider a physical system whose three-dimensional state space is spanned by an
	orthonormal basis $\{\ket{u_1},\ket{u_2},\ket{u_3}\}$. In that state space, consider two operators $L_z$ and S defined by:
	\[
		\begin{split}
		L_z\ket{u_1}=&\ket{u_1}, L_z\ket{u_2}=\ket{0}, L_z\ket{u_3}=-\ket{u_3} \\
		S\ket{u_1}=&\ket{u_3}, S\ket{u_2}=\ket{u_2}, S\ket{u_3}=\ket{u_1}
		\end{split}
	\]
	(a) Write the matrices which represent, in the $\{\ket{u_1},\ket{u_2},\ket{u_3}\}$ basis, the operators $L_z,L_z^2,S,$ and $S^2$. Are these operators observables? 
	\\
	\\
	(b) Give the form of the most general matrix, which represents an operator which
	commutes with $L_z$. Same for $L_z^2$ and $S^2$.
	\\
	\\
	(c) Do $L_z^2$ and $S^2$ form a CSCO? Give a basis of common eigenvectors.
	\\
	\\
	\textbf{Solution}
	\\
	\\
	\textbf{Part a}
	\\
	\\
	The matrix representation of these two operators is found by applying to each $\ket{u_i}$ and simply seeing what each row and column must be to bring about the given transformations. They are:
	\[
		L_z = \begin{pmatrix}
		1 & 0 & 0 \\
		0 & 0 & 0 \\
		0 & 0 &-1
		\end{pmatrix},
		S = \begin{pmatrix}
		0 & 0 & 1\\
		0 & 1 & 0 \\
		1 & 0 & 0
		\end{pmatrix}
	\]
	Then we square them to get the other two matrices:
	\[
		L_z^2 = \begin{pmatrix}
		1 & 0 & 0 \\
		0 & 0 & 0 \\
		0 & 0 & 1
		\end{pmatrix},
		S^2 = \begin{pmatrix}
		1 & 0 & 0\\
		0 & 1 & 0 \\
		0 & 0 & 1
		\end{pmatrix}
	\]
	These are all observables because they are Hermitian:
	\[
		\begin{split}
		L_z^\dagger =& L_z, (L_z^2)^\dagger=L_z^2 \\
		S^\dagger =& S, (S^2)^\dagger=S^2
		\end{split}
	\]
	\\
	\\
	\textbf{Part b}
	\\
	\\
	To find the most general matrix, A, which commutes with $L_z$, we need to solve:
	\[
		[L_z,A] = L_zA-AL_z = 0
	\]
	Let's define A in a general way as:
	\[
		A = \begin{pmatrix}
		a_{11} & a_{12} & a_{13} \\
		a_{21} & a_{22} & a_{23} \\
		a_{31} & a_{32} & a_{33}
		\end{pmatrix}
	\]
	Then we have:
	\[
	\begin{split}
		\begin{pmatrix}
		1 & 0 & 0 \\
		0 & 0 & 0 \\
		0 & 0 &-1
		\end{pmatrix}
		\begin{pmatrix}
		a_{11} & a_{12} & a_{13} \\
		a_{21} & a_{22} & a_{23} \\
		a_{31} & a_{32} & a_{33}
		\end{pmatrix} = &
		\begin{pmatrix}
		a_{11} & a_{12} & a_{13} \\
		a_{21} & a_{22} & a_{23} \\
		a_{31} & a_{32} & a_{33}
		\end{pmatrix}
		\begin{pmatrix}
		1 & 0 & 0 \\
		0 & 0 & 0 \\
		0 & 0 &-1
		\end{pmatrix}  \\
		\begin{pmatrix}
		a_{11} & a_{12} & a_{13} \\
		0 & 0 & 0 \\
		-a_{31} & -a_{32} & -a_{33}
		\end{pmatrix} =& 
		\begin{pmatrix}
		a_{11} & 0 & -a_{13} \\
		a_{21} & 0 & -a_{23} \\
		a_{31} & 0 & -a_{33}
		\end{pmatrix} \\
		\Rightarrow a_{11}=a_{11},\;a_{12}=0,\;a_{13}=-a_{13} \\
		a_{21}=0,\;a_{22}=a_{22},\;a_{23}=0 \\
		a_{31}=0,\;a_{23}=0,\;a_{33}=a_{33} \\
		\Rightarrow A = \begin{pmatrix}
		a_{11} & 0 & 0 \\
		0 & a_{22} & 0 \\
		0 & 0 & a_{33}
		\end{pmatrix}
	\end{split}
	\]
	Let's use the same general definition of A to find the general matrix which commutes with $L_z^2$:
	\[
		\begin{split}
		\begin{pmatrix}
		1 & 0 & 0 \\
		0 & 0 & 0 \\
		0 & 0 &1
		\end{pmatrix}
		\begin{pmatrix}
		a_{11} & a_{12} & a_{13} \\
		a_{21} & a_{22} & a_{23} \\
		a_{31} & a_{32} & a_{33}
		\end{pmatrix} = &
		\begin{pmatrix}
		a_{11} & a_{12} & a_{13} \\
		a_{21} & a_{22} & a_{23} \\
		a_{31} & a_{32} & a_{33}
		\end{pmatrix}
		\begin{pmatrix}
		1 & 0 & 0 \\
		0 & 0 & 0 \\
		0 & 0 & 1
		\end{pmatrix}  \\
		\begin{pmatrix}
		a_{11} & a_{12} & a_{13} \\
		0 & 0 & 0 \\
		a_{31} & a_{32} & a_{33}
		\end{pmatrix} =& 
		\begin{pmatrix}
		a_{11} & 0 & a_{13} \\
		a_{21} & 0 & a_{23} \\
		a_{31} & 0 & a_{33}
		\end{pmatrix} \\
		\Rightarrow a_{11}=a_{11},\;a_{12}=0,\;a_{13}=a_{13} \\
		a_{21}=0,\;a_{22}=a_{22},\;a_{23}=0 \\
		a_{31}=a_{31},\;a_{32}=0,\;a_{33}=a_{33} \\
		\Rightarrow A = \begin{pmatrix}
		a_{11} & 0 & a_{13} \\
		0 & a_{22} & 0 \\
		a_{31} & 0 & a_{33}
		\end{pmatrix} 
		\end{split}
	\]
	And now for $S^2$:
	\[
		\begin{split}
		\begin{pmatrix}
		1 & 0 & 0 \\
		0 & 1 & 0 \\
		0 & 0 &1
		\end{pmatrix}
		\begin{pmatrix}
		a_{11} & a_{12} & a_{13} \\
		a_{21} & a_{22} & a_{23} \\
		a_{31} & a_{32} & a_{33}
		\end{pmatrix} = &
		\begin{pmatrix}
		a_{11} & a_{12} & a_{13} \\
		a_{21} & a_{22} & a_{23} \\
		a_{31} & a_{32} & a_{33}
		\end{pmatrix}
		\begin{pmatrix}
		1 & 0 & 0 \\
		0 & 1 & 0 \\
		0 & 0 & 1
		\end{pmatrix}  \\
		\begin{pmatrix}
		a_{11} & a_{12} & a_{13} \\
		a_{21} & a_{22} & a_{23} \\
		a_{31} & a_{32} & a_{33}
		\end{pmatrix} =& 
		\begin{pmatrix}
		a_{11} & a_{12} & a_{13} \\
		a_{21} & a_{22} & a_{23} \\
		a_{31} & a_{32} & a_{33}
		\end{pmatrix} \\
		A = 
		\begin{pmatrix}
		a_{11} & a_{12} & a_{13} \\
		a_{21} & a_{22} & a_{23} \\
		a_{31} & a_{32} & a_{33}
		\end{pmatrix}
		\end{split}
	\]
	The commuting matrix for $L_z$ must be diagonal, for $L_z^2$ it must have the structure shown above, and for $S^2$ everything commutes because $S^2=I$, the identity.
	\\
	\\
	\textbf{Part c}
	\\
	\\
	The first thing to note is that both $L_Z^2$ and $S^2$ are degenerate. This means that the eigenvalues of either alone cannot fully specify the state of a vector. 
	\\
	
	Next, I want to see if $L_z^2$ commutes with $S^2$. If you refer back to the matrix representation of $S^2$ from part a, you'll see that $S^2=I$, where I is the identity matrix. \(\Rightarrow [S^2,L_z^2]=[I,L_Z^2]=0\). 
	\\
	
	But do they form a CSCO? We can quickly find the eigenvectors for $L_z^2$. Since the matrix is diagonal, we can simply read off the eigenvalues, $\lambda=1,0$. Plugging these into the characteristic equation, we get:
	\\
	For $\lambda=0$:
	\[
		\begin{split}
		\begin{pmatrix}
		1 & 0 & 0 \\
		0 & 0 & 0\\
		0 & 0 & 1
		\end{pmatrix}
		\begin{pmatrix} c_1 \\ c_2 \\ c_3 \end{pmatrix} =& \begin{pmatrix} 0\\0\\0 \end{pmatrix} \\
		\Rightarrow c_1 = 0, c_3 =& 0 \\
		\ket{v_{\lambda=0}} = \begin{pmatrix} 0 \\ 1 \\ 0 \end{pmatrix}
		\end{split}
	\]
	For $\lambda=1$:
	\[
		\begin{split}
		\begin{pmatrix}
		0 & 0 & 0 \\
		0 &-1 & 0\\
		0 & 0 & 0
		\end{pmatrix}
		\begin{pmatrix} c_1 \\ c_2 \\ c_3 \end{pmatrix} =& \begin{pmatrix} 0\\0\\0 \end{pmatrix} \\
		\Rightarrow c_2 =& 0 \\
		\ket{v_{\lambda=1}} = \begin{pmatrix} 1 \\ 0 \\ 1 \end{pmatrix}
		\end{split}
	\]
	But we can split $\ket{v_{\lambda=1}}$ into two orthogonal vectors:
	\[
		\ket{v_{\lambda=1}}=\begin{pmatrix} 1 \\ 0 \\ 0 \end{pmatrix} + \begin{pmatrix} 0 \\ 0 \\ 1 \end{pmatrix}
	\]
	Renaming the vectors as $v_1,v_2,v_3$:
	\[
		\ket{v_1}=\begin{pmatrix} 1 \\ 0 \\ 0 \end{pmatrix}, \ket{v_2}=\begin{pmatrix} 0 \\ 1 \\ 0 \end{pmatrix}, \ket{v_3}=\begin{pmatrix} 0 \\ 0 \\ 1 \end{pmatrix}
	\]
	Now we have three mutually orthogonal eigenvectors fully specifying the eigenspace. These are easy to check:
	\[
		\begin{split}
		L_z^2\ket{v_1} =& \ket{v_1}, \: S^2\ket{v_1} = \ket{v_1} \\
		L_z^2\ket{v_2} =& \ket{0}, \; S^2\ket{v_2} = \ket{v_2} \\
		L_z^2\ket{v_3} =& \ket{v_3}, \: S^2\ket{v_3} = \ket{v_3}
		\end{split}
	\]
	This has not gotten rid of our degeneracy. We still have $\lambda=1$ for $v_1$ and $v_3$. Thus, $L_z^2$ and $S^2$ do not form a CSCO.
	
\end{homeworkProblem}

\pagebreak

\begin{homeworkProblem}

\end{homeworkProblem}

\pagebreak

\begin{homeworkProblem}

\end{homeworkProblem}

\pagebreak

\begin{homeworkProblem}

\end{homeworkProblem}

\end{document}
