\documentclass{article}

\usepackage{hyperref}
\usepackage{fancyhdr}
\usepackage{braket}
\usepackage{extramarks}
\usepackage{amsmath}
\usepackage{amsthm}
\usepackage{amsfonts}
\usepackage{tikz}
\usepackage[plain]{algorithm}
\usepackage{algpseudocode}
\usepackage{mathtools}
\usepackage{graphicx}
\graphicspath{ {./Images/} }

\DeclarePairedDelimiter\abs{\lvert}{\rvert}%
\DeclarePairedDelimiter\norm{\lVert}{\rVert}%

\makeatletter
\let\oldabs\abs
\def\abs{\@ifstar{\oldabs}{\oldabs*}}
%
\let\oldnorm\norm
\def\norm{\@ifstar{\oldnorm}{\oldnorm*}}
\makeatother

\newcommand*{\Value}{\frac{1}{2}x^2}%

\usetikzlibrary{automata,positioning}

%
% Basic Document Settings
%

\topmargin=-0.45in
\evensidemargin=0in
\oddsidemargin=0in
\textwidth=6.5in
\textheight=9.0in
\headsep=0.25in

\linespread{1.1}

\pagestyle{fancy}
\lhead{\hmwkAuthorName}
\chead{\hmwkClass\ (\hmwkClassInstructor\ \hmwkClassTime): \hmwkTitle}
\rhead{\firstxmark}
\lfoot{\lastxmark}
\cfoot{\thepage}

\renewcommand\headrulewidth{0.4pt}
\renewcommand\footrulewidth{0.4pt}

\setlength\parindent{0pt}

%
% Create Problem Sections
%

\newcommand{\enterProblemHeader}[1]{
    \nobreak\extramarks{}{Problem \arabic{#1} continued on next page\ldots}\nobreak{}
    \nobreak\extramarks{Problem \arabic{#1} (continued)}{Problem \arabic{#1} continued on next page\ldots}\nobreak{}
}

\newcommand{\exitProblemHeader}[1]{
    \nobreak\extramarks{Problem \arabic{#1} (continued)}{Problem \arabic{#1} continued on next page\ldots}\nobreak{}
    \stepcounter{#1}
    \nobreak\extramarks{Problem \arabic{#1}}{}\nobreak{}
}

\setcounter{secnumdepth}{0}
\newcounter{partCounter}
\newcounter{homeworkProblemCounter}
\setcounter{homeworkProblemCounter}{1}
\nobreak\extramarks{Problem \arabic{homeworkProblemCounter}}{}\nobreak{}

%
% Homework Problem Environment
%
% This environment takes an optional argument. When given, it will adjust the
% problem counter. This is useful for when the problems given for your
% assignment aren't sequential. See the last 3 problems of this template for an
% example.
%
\newenvironment{homeworkProblem}[1][-1]{
    \ifnum#1>0
        \setcounter{homeworkProblemCounter}{#1}
    \fi
    \section{Problem \arabic{homeworkProblemCounter}}
    \setcounter{partCounter}{1}
    \enterProblemHeader{homeworkProblemCounter}
}{
    \exitProblemHeader{homeworkProblemCounter}
}

%
% Homework Details
%   - Title
%   - Due date
%   - Class
%   - Section/Time
%   - Instructor
%   - Author
%

\newcommand{\hmwkTitle}{Homework\ \#5}
\newcommand{\hmwkDueDate}{April , 2020}
\newcommand{\hmwkClass}{Physics 916}
\newcommand{\hmwkClassTime}{}
\newcommand{\hmwkClassInstructor}{Professor Jean Marcel Ngoko}
\newcommand{\hmwkAuthorName}{\textbf{Robert Tabb}}

%
% Title Page
%

\title{
    \vspace{2in}
    \textmd{\textbf{\hmwkClass:\ \hmwkTitle}}\\
    \normalsize\vspace{0.1in}\small{Due\ on\ \hmwkDueDate}\\
    \vspace{0.1in}\large{\textit{\hmwkClassInstructor\ \hmwkClassTime}}
    \vspace{3in}
}

\author{\hmwkAuthorName}
\date{}

\renewcommand{\part}[1]{\textbf{\large Part \Alph{partCounter}}\stepcounter{partCounter}\\}

%
% Various Helper Commands
%

% Useful for algorithms
\newcommand{\alg}[1]{\textsc{\bfseries \footnotesize #1}}

% For derivatives
\newcommand{\deriv}[1]{\frac{\mathrm{d}}{\mathrm{d}x} (#1)}

% For partial derivatives
\newcommand{\pderiv}[2]{\frac{\partial}{\partial #1} (#2)}

% Integral dx
\newcommand{\dx}{\mathrm{d}x}

% Alias for the Solution section header
\newcommand{\solution}{\textbf{\large Solution}}

% Probability commands: Expectation, Variance, Covariance, Bias
\newcommand{\E}{\mathrm{E}}
\newcommand{\Var}{\mathrm{Var}}
\newcommand{\Cov}{\mathrm{Cov}}
\newcommand{\Bias}{\mathrm{Bias}}

\begin{document}

\maketitle

\pagebreak

\begin{homeworkProblem}
	Show that in general, any $2\times2$ matrix $M$ can be represented in terms of the unit matrix, $I$, and the Pauli matrices. i.e.
	\[
		M=\begin{pmatrix} M_{11} & M_{12} \\ M_{21} & M_{22} \end{pmatrix} = a_0I+\vec{a}\cdot \vec{\sigma}
	\]
	where the expansion coefficients $a_i=\frac{1}{2}Tr\{M\sigma_i\}$
	\\
	\\
	\textbf{Solution}
	\\
	\\
	First I will use a common convention and define $\sigma_0$ as the identity operator. So we have:
	\[
		\begin{split} 
		M=\vec{a}\cdot\vec{\sigma} 
		=& a_0 \begin{pmatrix} 1 & 0 \\ 0 & 1 \end{pmatrix}
		+ a_1 \begin{pmatrix} 0 & 1 \\ 1 & 0 \end{pmatrix}
		+ a_2 \begin{pmatrix} 0 & -i \\ i & 0 \end{pmatrix}
		+ a_3 \begin{pmatrix} 1 & 0 \\ 0 & -1 \end{pmatrix} \\
		=& \begin{pmatrix} a_0+a_3 & a_1-ia_2 \\ a_1+ia_2 & a_0-a_3 \end{pmatrix} \\
		M_{11}=& a_0+a_3 \\
		M_{12}=& a_1-ia_2 \\
		M_{21}=& a_1+ia_2 \\
		M_{22}=& a_0-a_3 
		\end{split} 
	\]
	\[
		\begin{split}
		a_0=&\frac{1}{2}Tr\{ M\sigma_0 \} =\frac{1}{2}Tr\left[ \begin{pmatrix}M_{11} & M_{12} \\ M_{21} & M_{22} \end{pmatrix} \begin{pmatrix}1&0\\0&1\end{pmatrix} \right] = Tr\left[ \begin{pmatrix}M_{11} & M_{12} \\ M_{21} & M_{22} \end{pmatrix} \right]\\
		=& \frac{1}{2}\left( M_{11}+M_{22} \right) = \frac{1}{2}\left( a_0+a_3+a_0-a_3 \right) \\
		=& a_0 \\
		a_1=&\frac{1}{2}Tr\{ M\sigma_1 \} =\frac{1}{2}Tr\left[ \begin{pmatrix}M_{11} & M_{12} \\ M_{21} & M_{22} \end{pmatrix} \begin{pmatrix}0&1\\1&0\end{pmatrix} \right] = Tr\left[ \begin{pmatrix}M_{12} & M_{11} \\ M_{22} & M_{21} \end{pmatrix} \right]\\
		=& \frac{1}{2}\left( M_{12}+M_{21} \right) = \frac{1}{2}\left( a_1-ia_2+a_1+ia_2 \right) \\
		=& a_1 \\
		a_2=&\frac{1}{2}Tr\{ M\sigma_2 \} =\frac{1}{2}Tr\left[ \begin{pmatrix}M_{11} & M_{12} \\ M_{21} & M_{22} \end{pmatrix} \begin{pmatrix}0&-i\\i&0\end{pmatrix} \right] = Tr\left[ \begin{pmatrix}iM_{12} & -iM_{11} \\ iM_{22} & -iM_{21} \end{pmatrix} \right]\\
		=& \frac{1}{2}\left( iM_{12}-iM_{21} \right) = \frac{1}{2}\left( ia_1-i^2a_2-ia_1-i^2a_2 \right) \\
		=& a_2 \\
		a_3=&\frac{1}{2}Tr\{ M\sigma_3 \} =\frac{1}{2}Tr\left[ \begin{pmatrix}M_{11} & M_{12} \\ M_{21} & M_{22} \end{pmatrix} \begin{pmatrix}1&0\\0&-1\end{pmatrix} \right] = Tr\left[ \begin{pmatrix}M_{11} & -M_{12} \\ M_{21} & -M_{22} \end{pmatrix} \right]\\
		=& \frac{1}{2}\left( M_{11}-M_{22} \right) = \frac{1}{2}\left( a_0+a_3-a_0+a_3 \right) \\
		=& a_3
		\end{split} 
	\]
\end{homeworkProblem}

\pagebreak

\begin{homeworkProblem}
	Show that 
	\[
		\left[ S_x,S_y \right] = i\hbar S_z
	\]
	\textbf{Solution}
	\\
	\\
	\[
		S_x = \frac{\hbar}{2}\sigma_x,\; S_y = \frac{\hbar}{2}\sigma_y,\;S_z = \frac{\hbar}{2}\sigma_z
	\]
	\[
		\begin{split}
		S_xS_y=& \frac{\hbar^2}{4} \begin{pmatrix} 0 & 1 \\ 1 & 0 \end{pmatrix} 
		\begin{pmatrix} 0 & -i \\ i & 0 \end{pmatrix} = \frac{\hbar^2}{4} \begin{pmatrix} i & 0 \\ 0 & -i \end{pmatrix} \\
		S_yS_x=& \frac{\hbar^2}{4} \begin{pmatrix} 0 & -i \\ i & 0 \end{pmatrix}
		\begin{pmatrix} 0 & 1 \\ 1 & 0 \end{pmatrix} = \frac{\hbar^2}{4} \begin{pmatrix} -i & 0 \\ 0 & i \end{pmatrix} \\
		\end{split} 
	\]
	\[
		\begin{split} 
		S_xS_y-S_yS_x =& \frac{\hbar^2}{4} \begin{pmatrix} i & 0 \\ 0 & -i \end{pmatrix} - \frac{\hbar^2}{4} \begin{pmatrix} -i & 0 \\ 0 & i \end{pmatrix} \\
		=& \frac{\hbar^2}{4} \begin{pmatrix} 2i & 0 \\ 0 & -2i \end{pmatrix} = \frac{i\hbar^2}{2} \begin{pmatrix} 1 & 0 \\ 0 & -1 \end{pmatrix} \\
		=& i\hbar\left( \frac{\hbar}{2}\sigma_z \right) \\
		\Rightarrow \left[ S_x,S_y \right] =& i\hbar S_z
		\end{split} 
	\]
	
\end{homeworkProblem}
\end{document}
