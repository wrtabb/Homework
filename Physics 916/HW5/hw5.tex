\documentclass{article}

\usepackage{hyperref}
\usepackage{fancyhdr}
\usepackage{braket}
\usepackage{extramarks}
\usepackage{amsmath}
\usepackage{amsthm}
\usepackage{amsfonts}
\usepackage{tikz}
\usepackage[plain]{algorithm}
\usepackage{algpseudocode}
\usepackage{mathtools}
\usepackage{graphicx}
\graphicspath{ {./Images/} }

\DeclarePairedDelimiter\abs{\lvert}{\rvert}%
\DeclarePairedDelimiter\norm{\lVert}{\rVert}%

\makeatletter
\let\oldabs\abs
\def\abs{\@ifstar{\oldabs}{\oldabs*}}
%
\let\oldnorm\norm
\def\norm{\@ifstar{\oldnorm}{\oldnorm*}}
\makeatother

\newcommand*{\Value}{\frac{1}{2}x^2}%

\usetikzlibrary{automata,positioning}

%
% Basic Document Settings
%

\topmargin=-0.45in
\evensidemargin=0in
\oddsidemargin=0in
\textwidth=6.5in
\textheight=9.0in
\headsep=0.25in

\linespread{1.1}

\pagestyle{fancy}
\lhead{\hmwkAuthorName}
\chead{\hmwkClass\ (\hmwkClassInstructor\ \hmwkClassTime): \hmwkTitle}
\rhead{\firstxmark}
\lfoot{\lastxmark}
\cfoot{\thepage}

\renewcommand\headrulewidth{0.4pt}
\renewcommand\footrulewidth{0.4pt}

\setlength\parindent{0pt}

%
% Create Problem Sections
%

\newcommand{\enterProblemHeader}[1]{
    \nobreak\extramarks{}{Problem \arabic{#1} continued on next page\ldots}\nobreak{}
    \nobreak\extramarks{Problem \arabic{#1} (continued)}{Problem \arabic{#1} continued on next page\ldots}\nobreak{}
}

\newcommand{\exitProblemHeader}[1]{
    \nobreak\extramarks{Problem \arabic{#1} (continued)}{Problem \arabic{#1} continued on next page\ldots}\nobreak{}
    \stepcounter{#1}
    \nobreak\extramarks{Problem \arabic{#1}}{}\nobreak{}
}

\setcounter{secnumdepth}{0}
\newcounter{partCounter}
\newcounter{homeworkProblemCounter}
\setcounter{homeworkProblemCounter}{1}
\nobreak\extramarks{Problem \arabic{homeworkProblemCounter}}{}\nobreak{}

%
% Homework Problem Environment
%
% This environment takes an optional argument. When given, it will adjust the
% problem counter. This is useful for when the problems given for your
% assignment aren't sequential. See the last 3 problems of this template for an
% example.
%
\newenvironment{homeworkProblem}[1][-1]{
    \ifnum#1>0
        \setcounter{homeworkProblemCounter}{#1}
    \fi
    \section{Problem \arabic{homeworkProblemCounter}}
    \setcounter{partCounter}{1}
    \enterProblemHeader{homeworkProblemCounter}
}{
    \exitProblemHeader{homeworkProblemCounter}
}

%
% Homework Details
%   - Title
%   - Due date
%   - Class
%   - Section/Time
%   - Instructor
%   - Author
%

\newcommand{\hmwkTitle}{Homework\ \#5}
\newcommand{\hmwkDueDate}{May 4, 2020}
\newcommand{\hmwkClass}{Physics 916}
\newcommand{\hmwkClassTime}{}
\newcommand{\hmwkClassInstructor}{Professor Jean Marcel Ngoko}
\newcommand{\hmwkAuthorName}{\textbf{Robert Tabb}}

%
% Title Page
%

\title{
    \vspace{2in}
    \textmd{\textbf{\hmwkClass:\ \hmwkTitle}}\\
    \normalsize\vspace{0.1in}\small{Due\ on\ \hmwkDueDate}\\
    \vspace{0.1in}\large{\textit{\hmwkClassInstructor\ \hmwkClassTime}}
    \vspace{3in}
}

\author{\hmwkAuthorName}
\date{}

\renewcommand{\part}[1]{\textbf{\large Part \Alph{partCounter}}\stepcounter{partCounter}\\}

%
% Various Helper Commands
%

% Useful for algorithms
\newcommand{\alg}[1]{\textsc{\bfseries \footnotesize #1}}

% For derivatives
\newcommand{\deriv}[1]{\frac{\mathrm{d}}{\mathrm{d}x} (#1)}

% For partial derivatives
\newcommand{\pderiv}[2]{\frac{\partial}{\partial #1} (#2)}

% Integral dx
\newcommand{\dx}{\mathrm{d}x}

% Alias for the Solution section header
\newcommand{\solution}{\textbf{\large Solution}}

% Probability commands: Expectation, Variance, Covariance, Bias
\newcommand{\E}{\mathrm{E}}
\newcommand{\Var}{\mathrm{Var}}
\newcommand{\Cov}{\mathrm{Cov}}
\newcommand{\Bias}{\mathrm{Bias}}

\begin{document}

\maketitle

\pagebreak

\begin{homeworkProblem}
	Show that in general, any $2\times2$ matrix $M$ can be represented in terms of the unit matrix, $I$, and the Pauli matrices. i.e.
	\[
		M=\begin{pmatrix} M_{11} & M_{12} \\ M_{21} & M_{22} \end{pmatrix} = a_0I+\vec{a}\cdot \vec{\sigma}
	\]
	where the expansion coefficients $a_i=\frac{1}{2}Tr\{M\sigma_i\}$
	\\
	\\
	\textbf{Solution}
	\\
	\\
	First I will use a common convention and define $\sigma_0$ as the identity operator. So we have:
	\[
		\begin{split} 
		M=\vec{a}\cdot\vec{\sigma} 
		=& a_0 \begin{pmatrix} 1 & 0 \\ 0 & 1 \end{pmatrix}
		+ a_1 \begin{pmatrix} 0 & 1 \\ 1 & 0 \end{pmatrix}
		+ a_2 \begin{pmatrix} 0 & -i \\ i & 0 \end{pmatrix}
		+ a_3 \begin{pmatrix} 1 & 0 \\ 0 & -1 \end{pmatrix} \\
		=& \begin{pmatrix} a_0+a_3 & a_1-ia_2 \\ a_1+ia_2 & a_0-a_3 \end{pmatrix} \\
		M_{11}=& a_0+a_3 \\
		M_{12}=& a_1-ia_2 \\
		M_{21}=& a_1+ia_2 \\
		M_{22}=& a_0-a_3 
		\end{split} 
	\]
	\[
		\begin{split}
		a_0=&\frac{1}{2}Tr\{ M\sigma_0 \} =\frac{1}{2}Tr\left[ \begin{pmatrix}M_{11} & M_{12} \\ M_{21} & M_{22} \end{pmatrix} \begin{pmatrix}1&0\\0&1\end{pmatrix} \right] = Tr\left[ \begin{pmatrix}M_{11} & M_{12} \\ M_{21} & M_{22} \end{pmatrix} \right]\\
		=& \frac{1}{2}\left( M_{11}+M_{22} \right) = \frac{1}{2}\left( a_0+a_3+a_0-a_3 \right) \\
		=& a_0 \\
		a_1=&\frac{1}{2}Tr\{ M\sigma_1 \} =\frac{1}{2}Tr\left[ \begin{pmatrix}M_{11} & M_{12} \\ M_{21} & M_{22} \end{pmatrix} \begin{pmatrix}0&1\\1&0\end{pmatrix} \right] = Tr\left[ \begin{pmatrix}M_{12} & M_{11} \\ M_{22} & M_{21} \end{pmatrix} \right]\\
		=& \frac{1}{2}\left( M_{12}+M_{21} \right) = \frac{1}{2}\left( a_1-ia_2+a_1+ia_2 \right) \\
		=& a_1 \\
		a_2=&\frac{1}{2}Tr\{ M\sigma_2 \} =\frac{1}{2}Tr\left[ \begin{pmatrix}M_{11} & M_{12} \\ M_{21} & M_{22} \end{pmatrix} \begin{pmatrix}0&-i\\i&0\end{pmatrix} \right] = Tr\left[ \begin{pmatrix}iM_{12} & -iM_{11} \\ iM_{22} & -iM_{21} \end{pmatrix} \right]\\
		=& \frac{1}{2}\left( iM_{12}-iM_{21} \right) = \frac{1}{2}\left( ia_1-i^2a_2-ia_1-i^2a_2 \right) \\
		=& a_2 \\
		a_3=&\frac{1}{2}Tr\{ M\sigma_3 \} =\frac{1}{2}Tr\left[ \begin{pmatrix}M_{11} & M_{12} \\ M_{21} & M_{22} \end{pmatrix} \begin{pmatrix}1&0\\0&-1\end{pmatrix} \right] = Tr\left[ \begin{pmatrix}M_{11} & -M_{12} \\ M_{21} & -M_{22} \end{pmatrix} \right]\\
		=& \frac{1}{2}\left( M_{11}-M_{22} \right) = \frac{1}{2}\left( a_0+a_3-a_0+a_3 \right) \\
		=& a_3
		\end{split} 
	\]
\end{homeworkProblem}

\pagebreak
\begin{homeworkProblem}
	Consider the quantum operator, $H$, whose matrix representation in the orthonormal basis $\{ \ket{u_1},\ket{u_2} \}$ writes:
	\[
		H=\begin{pmatrix} H_{11} & H_{12} \\ H_{21} & H_{22} \end{pmatrix}
	\]
	where $H_{11}$ and $H_{22}$ are real numbers and $H_{12}=H_{21}^*$. It is thus obvious that $H$ is Hermitian. 
	\\
	\\
	1. Show that: \\
	\(H=\frac{1}{2}(H_{11}+H_{22})I+\tilde{K}\equiv\frac{1}{2}(H_{11}+H_{22})I+\frac{1}{2}(H_{11}-H_{22})K\) where $I$ is the identity operator, and the operators $\tilde{K}, K$ must be determined in terms of the matrix elements of $H$. Are $\tilde{K}$ and $K$ Hermitian?
	\\
	\\
	2. A key result from the decomposition in part 1 is that the operators $\tilde K, K$, and $H$ all have the same eigenvectors $\ket{\psi_\pm}$. Let $\tilde\kappa_\pm,\kappa_\pm,E_\pm$ be the eigenvalues of $\tilde K, K$, and $H$. Use the result of part 1 to establish the relation between $E_\pm$ and $\kappa_\pm$, and the relation between $E_\pm$ and $\tilde{\kappa}_\pm$. Show that these relations allow for a change of the eigenvalue origin.
	\\
	\\
	3. Directly solve the secular equations for $K$ and $H$ and determine the corresponding eigenvalues. Check that the relation between $E_\pm$ and $\kappa_\pm$ established in part 2 is correct.
	\\
	\\
	4. Let us define angles $0\leq\theta\leq\pi$ and $0\leq\phi\leq2\pi$ defined as:\\
	\(\tan\theta=\frac{2\abs{H_{21}}}{H_{11}-H_{22}}\) and \( H_{21}=\abs{H_{21}}e^{i\phi} \)
	\\
	\\
	5. Show that $E_++E_-=Tr\{H\}$, and that $E_+E_-=\det H$
	\\
	\\
	6. Show that if $H$ has a degenerate spectrum, then it is necessarily proportional to the identity operator.
	\\
	\\
	7. Use the operator, $K(\theta,\phi)$ to calculate normalized eigenvectors $\ket{\psi_\pm}$ in terms of these angles in the orthonormal basis $\{\ket{u_1},\ket{u_2}  \}$. You must find that the eigenvectors $\ket{\psi_\pm}$ are collinear to the eigenvectors $\ket{\pm}$ of the $1/2$ spin operator $S_u$, where $u$ is an arbitrary unit vector defined by these angles.
	\\
	\\
	8. Show that $K(\theta=0,\phi=0)$ is proportional to the z-component of the Pauli operator, $\sigma_z$. What are the corresponding eigenvalues and eigenvectors?
	\\
	\\
	9. When $\theta=\pi/2$, the operator $K$ is not finite and we must use $\tilde K$. Show that $\tilde K_x\equiv\tilde K(\theta=\pi/2,\phi=0)$ is proportional to the x-component of the Pauli operator, $\sigma_x$. What are the corresponding eigenvalues and eigenvectors?
	\\
	\\
	10. Show that $\tilde K_y \equiv\tilde K(\theta=\pi/2,\phi=\pi/2)$ is proportional to the y-component of the Pauli operator, $\sigma_y$. What are the corresponding eigenvalues and eigenvectors?
	\\
	\\
	11. Calculate the commutator, $[\tilde K_x, \tilde K_y]$, and show that it is proportional to the z-component of the pauli operator, $\sigma_z$.
	\\
	\\
	\textbf{Solution}
	\\
	\\
	Since they will be used several times later on in this problem, here are the definitions of the $2\times2$ Pauli operators:
	\[
		\sigma_x = \begin{pmatrix} 0 & 1 \\ 1 & 0 \end{pmatrix},\; 
		\sigma_y = \begin{pmatrix} 0 & -i \\ i & 0 \end{pmatrix},\;
		\sigma_z = \begin{pmatrix} 1 & 0 \\ 0 & -1 \end{pmatrix}
	\]
	1. If we let the following be true we will have the relation suggested:
	\[
		\begin{split} 
		H_{11}=&\frac{1}{2}[H_{11}(1+K_{11})+H_{22}(1-K_{11})] \\
		H_{22}=&\frac{1}{2}[H_{11}(1+K_{22})+H_{22}(1-K_{22})] \\
		H_{12}=&\frac{1}{2}K_{12}(H_{11}-H_{22})\\
		H_{21}=&\frac{1}{2}K_{12}(H_{11}-H_{22})
		\end{split} 
	\]
	From here we can determine a couple of things about $K$ and $\tilde{K}$
	\[
		\begin{split} 
		&K_{11}=1,\;K_{22}=-1 \\
		K_{12}=&\frac{2}{H_{11}-H_{22}}H_{12}=\frac{2}{H_{11}-H_{22}}H_{21}^*=K_{21}^*\\
		\end{split} 
	\]
	So we know the value of the diagonal of $K$ and $\tilde K$ and we also know that they are Hermitian.
	\\
	\\
	2. Starting with the relation between $E_\pm$ and $\tilde{\kappa}_\pm$
	\[
		\begin{split}
		H\ket{\psi_\pm}=&\frac{1}{2}(H_{11}+H_{22})I\ket{\psi_\pm}+\tilde K\ket{\psi_\pm} \\
		=& \frac{1}{2}(H_{11}+H_{22})\ket{\psi_\pm}+\tilde{\kappa}_\pm \ket{\psi_\pm} \\
		=& E_\pm\ket{\psi_\pm} \\
		\Rightarrow E_\pm=&\frac{1}{2}(H_{11}+H_{22})+\tilde{\kappa}_\pm
		\end{split} 
	\]
	Now for the relation between $E_\pm$ and $\kappa_\pm$
	\[
	\begin{split}
	H\ket{\psi_\pm}=&\frac{1}{2}(H_{11}+H_{22})I\ket{\psi_\pm}+ \frac{1}{2}(H_{11}-H_{22})K\ket{\psi_\pm} \\
	=& \frac{1}{2}(H_{11}+H_{22})\ket{\psi_\pm}+\frac{1}{2}(H_{11}-H_{22}){\kappa}_\pm \ket{\psi_\pm} \\
	=& E_\pm\ket{\psi_\pm} \\
	\Rightarrow E_\pm=&\frac{1}{2}(H_{11}+H_{22})+\frac{1}{2}(H_{11}-H_{22}){\kappa}_\pm
	\end{split} 
	\]
	In both of these cases, the eigenvalues are related by a shift by a constant, $\frac{1}{2}(H_{11}+H_{22})$, thus "shifting the origin" of the eigenvalues.
	\\
	\\
	3. Start by solving the secular equation to find the eigenvalues of $H$
	\[
		\begin{split}
		\begin{vmatrix} H_{11}-\lambda & H_{12} \\ H_{21} & H_{22}-\lambda \end{vmatrix} =& (H_{11}-\lambda)(H_{22}-\lambda)-\abs{H_{12}}^2 \\
		=& H_{11}H_{22}+\lambda^2-\lambda H_{22}-\lambda H_{11}-\abs{H_{12}}^2 \\
		=& \lambda^2-\lambda(H_{11}+H_{22})+H_{11}H_{22}-\abs{H_{12}}^2 \\
		=&0 \\
		\Rightarrow \lambda=&\frac{-b\pm\sqrt{b^2-4ac}}{2a} \\
		=&\frac{1}{2}(H_{11}+H_{22})\pm\frac{1}{2}\sqrt{(H_{11}+H_{22})^2-4(H_{11}H_{22}-\abs{H_{12}}^2)} \\
		=& \frac{1}{2}(H_{11}+H_{22})\pm\frac{1}{2}\sqrt{H_{11}^2+H_{22}^2+2H_{11}H_{22}-4H_{11}H_{22}+4\abs{H_{12}}^2} \\
		=& \frac{1}{2}(H_{11}+H_{22})\pm\frac{1}{2}\sqrt{H_{11}^2+H_{22}^2-2H_{11}H_{22}+4\abs{H_{12}}^2} \\
		=& \frac{1}{2}(H_{11}+H_{22})\pm\frac{1}{2}\sqrt{(H_{11}-H_{22})^2+4\abs{H_{12}}^2}
		\end{split} 
	\]
	Now I'll do the same for the eigenvalues of $K$. Recall from part 1 that $K_{11} = 1,\;K_{22}=-1$
	\[
		\begin{split} 
		\begin{vmatrix} 1-\lambda & K_{12} \\ K_{21} & -(1+\lambda) \end{vmatrix} =& -(1+\lambda)(1-\lambda) -\abs{K_{12}}^2 \\
		=& \lambda^2-1-\abs{K_{12}}^2 \\
		=&0 \\
		\Rightarrow \lambda =& \pm\sqrt{1+\abs{K_{12}}^2}
		\end{split}
	\]
	Recall from part 1: $K_{12}=\frac{2}{H_{11}-H_{22}}H_{12}$
	\[
		\begin{split}
		 \lambda =& \pm\sqrt{1+\abs{\frac{2}{H_{11}-H_{22}}H_{12}}^2} \\
		 =& \pm\frac{1}{H_{11}-H_{22}}\sqrt{(H_{11}-H_{22})^2+4\abs{H_{12}}^2} \\
		\end{split}  
	\]
	4. We have:
	\[
		\begin{split} 
		\tan\theta=&\frac{2\abs{H_{21}}}{H_{11}-H_{22}},\; H_{21}=\abs{H_{21}}e^{i\phi} \\
		\Rightarrow e^{i\phi}\tan\theta=&\frac{2H_{21}}{H_{11}-H_{22}} \\
		K_{21} =& e^{i\phi}\tan\theta \\
		K_{12} =&e^{-i\phi}\tan\theta
		\end{split} 
	\]
	We already established that $K_{11}=1$ and $K_{22}=-1$, so we have:
	\[
		K=\begin{pmatrix} 1 & e^{-i\phi}\tan\theta \\ e^{i\phi}\tan\theta & -1 \end{pmatrix}
	\]
	Then the eigenvalue can be expressed using the equation in the last part of part 3:
	\[
		\begin{split} 
		\kappa_\pm =&\pm\sqrt{1+\abs{\frac{2}{H_{11}-H_{22}}H_{12}}^2} \\
		=& \pm\sqrt{1+\abs{e^{i\phi}\tan\theta}^2} \\
		=& \pm\sqrt{1+\tan^2\theta} \\
		=& \pm\sec\theta
		\end{split} 
	\]
	5. Taking the eigenvalues of $H$ from part 3:
	\[
		\begin{split} 
		E_++E_- =& \frac{1}{2}(H_{11}+H_{22})+\frac{1}{2}\sqrt{(H_{11}-H_{22})^2+4\abs{H_{12}}^2} + \frac{1}{2}(H_{11}+H_{22})-\frac{1}{2}\sqrt{(H_{11}-H_{22})^2+4\abs{H_{12}}^2} \\
		=& \frac{1}{2}(H_{11}+H_{22})+\frac{1}{2}(H_{11}+H_{22}) \\
		=& H_{11}+H_{22} \\
		=& Tr\{H\}
		\end{split} 
	\]
	Now to calculate $E_+E_-$. The form of this will be $(a+b)(a-b)=a^2-b^2$:
	\[
		\begin{split}
		E_+E_-=& \left( \frac{1}{2}(H_{11}+H_{22}) \right)^2-\left( \frac{1}{2}\sqrt{(H_{11}-H_{22})^2+4\abs{H_{12}}^2} \right)^2 \\
		=& \frac{1}{4}\left(H_{11}^2+H_{22}^2+2H_{11}H_{22}\right)-\frac{1}{4}(H_{11}-H_{22})^2-\abs{H_{12}}^2 \\
		=& \frac{1}{4}\left(H_{11}^2+H_{22}^2+2H_{11}H_{22}\right)-\frac{1}{4}(H_{11}^2+H_{22}^2-2H_{11}H_{22})-\abs{H_{12}}^2 \\
		=&\frac{1}{4}2H_{11}H_{22}-\frac{1}{4}(-2H_{11}H_{22})-\abs{H_{12}}^2 \\
		=& H_{11}H_{22}-\abs{H_{12}}^2 \\
		=& \det H
		\end{split} 
	\]
	6. Since $H$ is only $2\times2$, there are only two eigenvalues, so for it to be degenerate means that the eigenvalues are the same. $\Rightarrow E_+=E_-$
	\[
		\begin{split} 
		\Rightarrow &\frac{1}{2}(H_{11}+H_{22})+\frac{1}{2}\sqrt{(H_{11}-H_{22})^2+4\abs{H_{12}}^2}=\frac{1}{2}(H_{11}+H_{22})-\frac{1}{2}\sqrt{(H_{11}-H_{22})^2+4\abs{H_{12}}^2} \\
		&(H_{11}-H_{22})^2+4\abs{H_{12}}^2=0
		\end{split}
	\]
	For this to be true, $\abs{H_{12}}=0$. This is because otherwise $H_{11}$ and/or $H_{22}$ would have to be complex, and it was specified in the problem that they are real. Then if $\abs{H_{12}}=0$, $H_{11}=H_{22}$. This means that $H$ must be proportional to the identity matrix.
	\\
	\\
	7. The eigenvalues of $K$ are $\kappa_\pm=\pm\sec\theta$, and $K$ is defined as:
	\[
		K=\begin{pmatrix} 1 & e^{-i\phi}\tan\theta \\ e^{i\phi}\tan\theta & -1 \end{pmatrix}
	\]
	To find the eigenvectors, starting with $\kappa_+$
	\[
		\begin{split}
		&\begin{pmatrix} 1-\sec\theta & e^{-i\phi}\tan\theta \\ e^{i\phi}\tan\theta & -(1+\sec\theta) \end{pmatrix} \begin{pmatrix} c_1 \\ c_2 \end{pmatrix} = \begin{pmatrix} 0 \\ 0 \end{pmatrix} \\
		&\Rightarrow c_2 = c_1\frac{\sec\theta-1}{\tan\theta}e^{i\phi}=c_1\tan\frac{\theta}{2}e^{i\phi} \\
		&v_+ = \begin{pmatrix} 1 \\ \tan\frac{\theta}{2}e^{i\phi} \end{pmatrix}
		\end{split} 
	\]
	Now to normalize it:
	\[
		\begin{split}
		&A_+^2(1+\tan^2\frac{\theta}{2}) = 1 \\
		&A_+=\sqrt{\frac{1}{1+\tan^2\frac{\theta}{2}}} = \sqrt{\frac{1}{\sec^2\frac{\theta}{2}}} = \cos\frac{\theta}{2}
		\end{split}  
	\]
	For the eigenvalue $\kappa_+$, we have:
	\[
		\ket{\psi_+} = \cos\frac{\theta}{2}\begin{pmatrix} 1 \\ \tan\frac{\theta}{2}e^{i\phi} \end{pmatrix}
	\]
	
	Now for $\kappa_-$
	\[
	\begin{split}
	&\begin{pmatrix} \sec\theta+1 & e^{-i\phi}\tan\theta \\ e^{i\phi}\tan\theta & \sec\theta-1 \end{pmatrix} \begin{pmatrix} c_1 \\ c_2 \end{pmatrix} = \begin{pmatrix} 0 \\ 0 \end{pmatrix} \\
	&\Rightarrow c_2 = c_1\frac{\tan\theta}{1-\sec\theta}e^{i\phi}=-c_1\cot\frac{\theta}{2}e^{i\phi} \\
	&v_- = \begin{pmatrix} 1 \\ -\cot\frac{\theta}{2}e^{i\phi} \end{pmatrix}
	\end{split} 
	\]
	Now to normalize it:
	\[
	\begin{split}
	&A_-^2(1+\cot^2\frac{\theta}{2}) = 1 \\
	&A_-=\sqrt{\frac{1}{1+\cot^2\frac{\theta}{2}}} = \sqrt{\frac{1}{\csc^2\frac{\theta}{2}}} = \sin\frac{\theta}{2}
	\end{split}  
	\]
	For the eigenvalue $\kappa_+$, we have:
	\[
	\ket{\psi_-} = \sin\frac{\theta}{2}\begin{pmatrix} 1 \\ -\cot\frac{\theta}{2}e^{i\phi} \end{pmatrix}
	\]
	8. The operator $K(\theta,\phi)$ is defined in the matrix representation as:
	\[
		 K=\begin{pmatrix} 1 & e^{-i\phi}\tan\theta \\ e^{i\phi}\tan\theta & -1 \end{pmatrix}
	\]
	If we let $\theta=\phi=0$, we have:
	\[
		K=\begin{pmatrix} 1 & 0 \\ 0 & -1 \end{pmatrix}
	\]
	This is equal to the Pauli matrix, $\sigma_z$. The eigenvalues and eigenvectors can be found by solving the characteristic equation:
	\[
		\begin{split}
		&\begin{vmatrix} 1-\lambda & 0 \\ 0 &-(1+\lambda) \end{vmatrix} = 0 \\
		&\lambda = \pm 1 \\
		\\
		&\lambda=1 \\
		&\begin{pmatrix} 0 & 0 \\ 0 &-2 \end{pmatrix} \begin{pmatrix} c_1 \\ c_2 \end{pmatrix}= \begin{pmatrix} 0 \\ 0 \end{pmatrix} \\
		&c_2=0 \\
		&\Rightarrow v_1 = \begin{pmatrix} 1 \\ 0 \end{pmatrix} \\
		\\
		&\lambda=-1 \\
		&\begin{pmatrix} 2 & 0 \\ 0 & 0 \end{pmatrix} \begin{pmatrix} c_1 \\ c_2 \end{pmatrix}= \begin{pmatrix} 0 \\ 0 \end{pmatrix} \\
		&c_1=0 \\
		&\Rightarrow v_1 = \begin{pmatrix} 0 \\ 1 \end{pmatrix} 
		\end{split}  
	\]
	\\
	\\
	9. Using the definitions from part 4, we can write:
	\[
		H_{11}-H_{22}=\frac{2\abs{H_{12}}}{\tan\theta}
	\]
	\[
		\begin{split}
		\tilde K=\frac{1}{2}(H_{11}-H_{22})K =& \begin{pmatrix} \frac{\abs{H_{12}}}{\tan\theta} & e^{-i\phi}\tan\theta\frac{\abs{H_{12}}}{\tan\theta} \\ e^{i\phi}\tan\theta\frac{\abs{H_{12}}}{\tan\theta} & -\frac{\abs{H_{12}}}{\tan\theta} \end{pmatrix} \\
		=& \begin{pmatrix} \frac{\abs{H_{12}}}{\tan\theta} & \abs{H_{12}}e^{-i\phi} \\ \abs{H_{12}}e^{i\phi} & -\frac{\abs{H_{12}}}{\tan\theta} \end{pmatrix}
		\end{split}
	\]
	For $\theta=\pi/2$ and $\phi=0$, we have:
	\[
		\begin{split}
		\tilde K_x
		=& \abs{H_{12}}\begin{pmatrix} 0 & 1 \\ 1 & 0 \end{pmatrix}\propto \sigma_x
		\end{split}
	\]
	Now we use the usual method to find eigenvalues and eigenvectors:
	\[
		\begin{split}
		&\begin{vmatrix} -\lambda & 1 \\ 1 &-\lambda \end{vmatrix} = \lambda^2-1 = 0 \\
		&\lambda = \pm 1 \\
		\\
		&\lambda=1 \\
		&\begin{pmatrix} -1 & 1 \\ 1 & -1 \end{pmatrix} \begin{pmatrix} c_1 \\ c_2 \end{pmatrix}= \begin{pmatrix} 0 \\ 0 \end{pmatrix} \\
		&c_2=c_1 \\
		&\Rightarrow v_1 = \frac{1}{\sqrt2}\begin{pmatrix} 1 \\ 1 \end{pmatrix} \\
		\\
		&\lambda=-1 \\
		&\begin{pmatrix} 1 & 1 \\ 1 & 1 \end{pmatrix} \begin{pmatrix} c_1 \\ c_2 \end{pmatrix}= \begin{pmatrix} 0 \\ 0 \end{pmatrix} \\
		&c_2=-c_1 \\
		&\Rightarrow v_1 = \frac{1}{\sqrt2}\begin{pmatrix} 1 \\ -1 \end{pmatrix} 
		\end{split}
	\]
	10. 
	\[
		\begin{split}
		\tilde K=& \begin{pmatrix} \frac{\abs{H_{12}}}{\tan\theta} & \abs{H_{12}}e^{-i\phi} \\ \abs{H_{12}}e^{i\phi} & -\frac{\abs{H_{12}}}{\tan\theta} \end{pmatrix} \\
		\tilde K(\pi/2,\pi/2)=& \begin{pmatrix} 0 & \abs{H_{12}}e^{-i\pi/2} \\ \abs{H_{12}}e^{i\pi/2} & 0 \end{pmatrix}
		\end{split} 
	\]
	\[
		\begin{split}
		e^{i\pi/2}=&\cos(\pi/2)+i\sin(\pi/2) = i \\
		e^{-i\pi/2}=&\cos(\pi/2)-i\sin(\pi/2) = -i \\
		\end{split}
	\]
	\[
		\Rightarrow \tilde K(\pi/2,\pi/2)= \abs{H_{12}}\begin{pmatrix} 0 & -i \\ i & 0 \end{pmatrix} \propto \sigma_y
	\]
		Now we use the usual method to find eigenvalues and eigenvectors:
		\[
		\begin{split}
		&\begin{vmatrix} -\lambda & -i \\ i &-\lambda \end{vmatrix} = \lambda^2+i^2 = 0 \\
		&\lambda = \pm 1 \\
		\\
		&\lambda=1 \\
		&\begin{pmatrix} -1 & -i \\ i & -1 \end{pmatrix} \begin{pmatrix} c_1 \\ c_2 \end{pmatrix}= \begin{pmatrix} 0 \\ 0 \end{pmatrix} \\
		&c_2=ic_1 \\
		&\Rightarrow v_1 = \frac{1}{\sqrt2}\begin{pmatrix} 1 \\ i \end{pmatrix} \\
		\\
		&\lambda=-1 \\
		&\begin{pmatrix} 1 & -i \\ i & 1 \end{pmatrix} \begin{pmatrix} c_1 \\ c_2 \end{pmatrix}= \begin{pmatrix} 0 \\ 0 \end{pmatrix} \\
		&c_2=-ic_1 \\
		&\Rightarrow v_1 = \frac{1}{\sqrt2}\begin{pmatrix} 1 \\ -i \end{pmatrix} 
		\end{split}
		\]
		
		11. 
		\[
			\begin{split}
			[\tilde K_x, \tilde K_y] =& \tilde K_x\tilde K_y - \tilde K_y\tilde K_x \\
			\tilde K_x\tilde K_y =&  \abs{H_{12}}^2\begin{pmatrix} 0 & 1 \\ 1 & 0 \end{pmatrix}\begin{pmatrix} 0 & -i \\ i & 0 \end{pmatrix} = 
			\abs{H_{12}}^2\begin{pmatrix} i & 0 \\ 0 & -i \end{pmatrix} \\
			\tilde K_y\tilde K_x =&  \abs{H_{12}}^2\begin{pmatrix} 0 & -i \\ i & 0 \end{pmatrix}\begin{pmatrix} 0 & 1 \\ 1 & 0 \end{pmatrix} =
			\abs{H_{12}}^2\begin{pmatrix} -i & 0 \\ 0 & i \end{pmatrix} \\
			[\tilde K_x, \tilde K_y] =& \abs{H_{12}}^2\begin{pmatrix} i & 0 \\ 0 & -i \end{pmatrix} - \abs{H_{12}}^2\begin{pmatrix} -i & 0 \\ 0 & i \end{pmatrix} \\
			=& \abs{H_{12}}^2 \begin{pmatrix} 2i & 0 \\ 0 & -2i \end{pmatrix} \\
			=& 2i\abs{H_{12}}^2 \begin{pmatrix} 1 & 0 \\ 0 & -1 \end{pmatrix} \propto \sigma_z
			\end{split}
		\]
\end{homeworkProblem}

\pagebreak

\begin{homeworkProblem}
	The spin operator, $\boldsymbol{S}$ of an electron is pointing in any direction and is related to the Pauli matrices as $\boldsymbol{S}=\frac{\hbar}{2}\boldsymbol{\sigma}$ in the orthonormal basis $\{\ket{+},\ket{-}\}$ for $S_z$.
	\\
	\\
	1. Write down the matrix for $S_x,\;S_y,\;S_z,\;S_u$. Are they Hermitian? \\
	2. Determine the eigenvalues of each component for the spin operator. \\
	3. Determine the eigenvectors of each component for the spin operator. \\
	4. Show that $[S_x,S_y]=i\hbar S_z$, $[S_y,S_z]=i\hbar S_x$, $[S_z,S_x]=i\hbar S_y$ \\
	5. Show that $[\boldsymbol{S}^2,\boldsymbol{S}]=0$ 
	\\
	\\

	\textbf{Solution}
	\\
	\\
	1. For $S_x,S_y,S_z$, we simply plug in the Pauli matrices.
	\[
		\begin{split} 
		S_x=& \frac{\hbar}{2}\sigma_x = \frac{\hbar}{2}\begin{pmatrix} 0 & 1 \\ 1 & 0 \end{pmatrix} \\
		S_y=& \frac{\hbar}{2}\sigma_y = \frac{\hbar}{2}\begin{pmatrix} 0 & -i \\ i & 0 \end{pmatrix} \\
		S_z=& \frac{\hbar}{2}\sigma_z = \frac{\hbar}{2}\begin{pmatrix} 1 & 0 \\ 0 & -1 \end{pmatrix}
		\end{split} 
	\]
	To find $S_u$ we use the definition: $S_u=\boldsymbol{S}\cdot\boldsymbol{u}$, where $\boldsymbol{u}$ is the unit vector in 3-dimensional space.
	\[
		\begin{split}
		S_u =& \boldsymbol{S}\cdot\boldsymbol{u}=S_xu_x+S_yu_y+S_zu_z \\
		=&\frac{\hbar}{2}\begin{pmatrix} 0 & 1 \\ 1 & 0 \end{pmatrix} \sin\theta\cos\phi
		+\frac{\hbar}{2}\begin{pmatrix} 0 & -i \\ i & 0 \end{pmatrix} \sin\theta\sin\phi
		+\frac{\hbar}{2}\begin{pmatrix} 1 & 0 \\ 0 & -1 \end{pmatrix} \cos\theta \\
		=& \frac{\hbar}{2}\begin{pmatrix} \cos\theta & \sin\theta\cos\phi-i\sin\theta\sin\phi \\ \sin\theta\cos\phi+i\sin\theta\sin\phi & \cos\theta \end{pmatrix} \\
		=& \frac{\hbar}{2}\begin{pmatrix} \cos\theta & \sin\theta(\cos\phi-i\sin\phi) \\ \sin\theta(\cos\phi+i\sin\phi) & \cos\theta \end{pmatrix} \\
		=& \frac{\hbar}{2}\begin{pmatrix} \cos\theta & e^{-i\phi}\sin\theta \\ e^{i\phi}\sin\theta & \cos\theta \end{pmatrix}
		\end{split}
	\]
	Each of these matrices is symmetric except for the two cases with complex parts, $S_y,S_u$, but in both of those matrices, $S_{21} = S_{12}^*$, so these are all Hermitian.
	\\
	\\
	2. We find the eigenvalues in the usual way by solving the characteristic equation. \\
	For $S_x$:
	\[
		\begin{split} 
		\begin{vmatrix} -\lambda & \hbar/2 \\ \hbar/2 & -\lambda \end{vmatrix} = \lambda^2-\frac{\hbar^2}{4} =0 \Rightarrow \lambda = \pm \hbar/2 \\
		\end{split}
	\]
	For $S_y$:
	\[
	\begin{split} 
	\begin{vmatrix} -\lambda & -i\hbar/2 \\ i\hbar/2 & -\lambda \end{vmatrix} = \lambda^2+i^2\frac{\hbar^2}{4} = \lambda^2-\frac{\hbar^2}{4} = 0 \Rightarrow \lambda = \pm \hbar/2 \\
	\end{split}
	\]
	For $S_x$:
	\[
	\begin{split} 
	\begin{vmatrix} \hbar/2-\lambda & 0 \\ 0 & \hbar/2-\lambda \end{vmatrix} = -\left(\frac{\hbar}{2}+\lambda\right)\left(\frac{\hbar}{2}-\lambda\right) = 0 \Rightarrow \lambda = \pm \hbar/2
	\end{split}
	\]
	So each of the spin operators has the same eigenvalues.
	\\
	\\
	3. For each of the spin operators, plug the eigenvalues calculated in the last part into the eigenvalue equation and calculate the eigenvectors.
	\\
	For $S_x$,$\lambda=+1$ (leaving off the factors of $\hbar/2$ since they don't matter here):
	\[
		\begin{split} 
		&\begin{pmatrix} -1 & 1 \\ 1 & -1 \end{pmatrix} \begin{pmatrix} c_1 \\ c_2 \end{pmatrix} = \begin{pmatrix} 0 \\ 0 \end{pmatrix} \\
		&c_2=c_1 \\
		&\vec{v}_{\lambda=+1} = \frac{1}{\sqrt{2}}\begin{pmatrix} 1 \\ 1 \end{pmatrix}
		\end{split}
	\]
	For $S_x$,$\lambda=-1$:
	\[
	\begin{split} 
	&\begin{pmatrix} 1 & 1 \\ 1 & 1 \end{pmatrix} \begin{pmatrix} c_1 \\ c_2 \end{pmatrix} = \begin{pmatrix} 0 \\ 0 \end{pmatrix} \\
	&c_2=-c_1 \\
	&\vec{v}_{\lambda=-1} = \frac{1}{\sqrt{2}}\begin{pmatrix} 1 \\ -1 \end{pmatrix}
	\end{split}
	\]
	For $S_y$,$\lambda=+1$:
	\[
	\begin{split} 
	&\begin{pmatrix} -1 & -i \\ i & -1 \end{pmatrix} \begin{pmatrix} c_1 \\ c_2 \end{pmatrix} = \begin{pmatrix} 0 \\ 0 \end{pmatrix} \\
	&c_2=ic_1 \\
	&\vec{v}_{\lambda=+1} = \frac{1}{\sqrt{2}}\begin{pmatrix} 1 \\ i \end{pmatrix}
	\end{split}
	\]
	For $S_y$,$\lambda=-1$:
	\[
	\begin{split} 
	&\begin{pmatrix} 1 & -i \\ i & 1 \end{pmatrix} \begin{pmatrix} c_1 \\ c_2 \end{pmatrix} = \begin{pmatrix} 0 \\ 0 \end{pmatrix} \\
	&c_2=-ic_1 \\
	&\vec{v}_{\lambda=-1} = \frac{1}{\sqrt{2}}\begin{pmatrix} 1 \\ -i \end{pmatrix}
	\end{split}
	\]
	For $S_z$,$\lambda=+1$:
	\[
	\begin{split} 
	&\begin{pmatrix} 0 & 0 \\ 0 & -2 \end{pmatrix} \begin{pmatrix} c_1 \\ c_2 \end{pmatrix} = \begin{pmatrix} 0 \\ 0 \end{pmatrix} \\
	&c_2=0 \\
	&\vec{v}_{\lambda=+1} = \begin{pmatrix} 1 \\ 0 \end{pmatrix}
	\end{split}
	\]
	For $S_z$,$\lambda=-1$:
	\[
	\begin{split} 
	&\begin{pmatrix} 2 & 0 \\ 0 & 0 \end{pmatrix} \begin{pmatrix} c_1 \\ c_2 \end{pmatrix} = \begin{pmatrix} 0 \\ 0 \end{pmatrix} \\
	&c_1=0 \\
	&\vec{v}_{\lambda=+1} = \begin{pmatrix} 0 \\ 1 \end{pmatrix}
	\end{split}
	\]
	

	4. Use the matrix definitions given in part 1 to explicitly calculate the commutators
	\[
		\begin{split}
		S_xS_y=& \frac{\hbar^2}{4} \begin{pmatrix} 0 & 1 \\ 1 & 0 \end{pmatrix} 
		\begin{pmatrix} 0 & -i \\ i & 0 \end{pmatrix} = \frac{\hbar^2}{4} \begin{pmatrix} i & 0 \\ 0 & -i \end{pmatrix} \\
		S_yS_x=& \frac{\hbar^2}{4} \begin{pmatrix} 0 & -i \\ i & 0 \end{pmatrix}
		\begin{pmatrix} 0 & 1 \\ 1 & 0 \end{pmatrix} = \frac{\hbar^2}{4} \begin{pmatrix} -i & 0 \\ 0 & i \end{pmatrix} \\
		\end{split} 
	\]
	\[
		\begin{split} 
		S_xS_y-S_yS_x =& \frac{\hbar^2}{4} \begin{pmatrix} i & 0 \\ 0 & -i \end{pmatrix} - \frac{\hbar^2}{4} \begin{pmatrix} -i & 0 \\ 0 & i \end{pmatrix} \\
		=& \frac{\hbar^2}{4} \begin{pmatrix} 2i & 0 \\ 0 & -2i \end{pmatrix} = \frac{i\hbar^2}{2} \begin{pmatrix} 1 & 0 \\ 0 & -1 \end{pmatrix} \\
		=& i\hbar\left( \frac{\hbar}{2}\sigma_z \right) \\
		\Rightarrow \left[ S_x,S_y \right] =& i\hbar S_z \\
		\end{split}
	\]
	
		\[
		\begin{split}
		S_yS_z=& \frac{\hbar^2}{4} \begin{pmatrix} 0 & -i \\ i & 0 \end{pmatrix} 
		\begin{pmatrix} 1 & 0 \\ 0 & -1 \end{pmatrix} = \frac{\hbar^2}{4} \begin{pmatrix} 0 & i \\ i & 0 \end{pmatrix} \\
		S_zS_y=& \frac{\hbar^2}{4} \begin{pmatrix} 1 & 0 \\ 0 & -1 \end{pmatrix}\begin{pmatrix} 0 & -i \\ i & 0 \end{pmatrix}
		 = \frac{\hbar^2}{4} \begin{pmatrix} 0 & -i \\ -i & 0 \end{pmatrix} \\
		\end{split} 
		\]
		\[
		\begin{split} 
		S_yS_z-S_zS_y=& \frac{\hbar^2}{4} \begin{pmatrix} 0 & i \\ i & 0 \end{pmatrix} - \frac{\hbar^2}{4} \begin{pmatrix} 0 & -i \\ -i & 0 \end{pmatrix} \\
		=& \frac{\hbar^2}{4} \begin{pmatrix} 0 & 2i \\ 2i & 0 \end{pmatrix} = \frac{i\hbar^2}{2} \begin{pmatrix} 0 & 1 \\ 1 & 0 \end{pmatrix} \\
		=&i\hbar\left( \frac{\hbar}{2}\sigma_x \right) \\
		\Rightarrow \left[ S_y,S_z \right] =& i\hbar S_x
		\end{split}
		\]
		
		\[
		\begin{split}
		S_zS_x=& \frac{\hbar^2}{4} \begin{pmatrix} 1 & 0 \\ 0 & -1 \end{pmatrix} 
		\begin{pmatrix} 0 & 1 \\ 1 & 0 \end{pmatrix} = \frac{\hbar^2}{4} \begin{pmatrix} 0 & 1 \\ -1 & 0 \end{pmatrix} \\
		S_xS_z=& \begin{pmatrix} 0 & 1 \\ 1 & 0 \end{pmatrix}\begin{pmatrix} 1 & 0 \\ 0 & -1 \end{pmatrix}
		= \frac{\hbar^2}{4} \begin{pmatrix} 0 & -1 \\ 1 & 0 \end{pmatrix} \\
		\end{split} 
		\]
		\[
		\begin{split} 
		S_zS_x-S_xS_z=& \frac{\hbar^2}{4} \begin{pmatrix} 0 & 1 \\ -1 & 0 \end{pmatrix}-\frac{\hbar^2}{4} \begin{pmatrix} 0 & -1 \\ 1 & 0 \end{pmatrix}\\
		=& \frac{\hbar^2}{4} \begin{pmatrix} 0 & 2 \\ -2 & 0 \end{pmatrix}=\frac{i\hbar^2}{2} \begin{pmatrix} 0 & -i \\ i & 0 \end{pmatrix} \\
		=& i\hbar\left( \frac{\hbar}{2}\sigma_y \right) \\
		\Rightarrow \left[ S_z,S_x \right] =& i\hbar S_y
		\end{split}
		\]
		5. To start with, calculate $\boldsymbol{S}^2$
		\[
			\begin{split}
			\boldsymbol{S}=&\begin{pmatrix} S_x \\ S_y\\ S_z \end{pmatrix} = \frac{\hbar}{2} \begin{pmatrix}\sigma_x\\\sigma_y\\\sigma_z\end{pmatrix} \\
			\boldsymbol{S}^2=&\frac{\hbar^2}{4} \begin{pmatrix}\sigma_x&\sigma_y&\sigma_z\end{pmatrix} \begin{pmatrix}\sigma_x\\\sigma_y\\\sigma_z\end{pmatrix} \\
			=& \frac{\hbar^2}{4}[\sigma_x^2+\sigma_y^2+\sigma_z^2] \\
			=& \frac{\hbar^2}{4} [I+I+I] \\
			=& \frac{3\hbar^2}{4} I
			\end{split} 
		\]
		For each of the Pauli matrices, $\sigma_i^2=I$, the identity matrix. So now we know:
		\[
			\left[ \boldsymbol{S}^2,\boldsymbol{S} \right]\propto \left[ I,\boldsymbol{S} \right]=0
		\]
		Because the identity matrix commutes with everything.
\end{homeworkProblem}
\end{document}
