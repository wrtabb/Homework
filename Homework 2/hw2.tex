\documentclass{article}

\usepackage{fancyhdr}
\usepackage{extramarks}
\usepackage{amsmath}
\usepackage{amsthm}
\usepackage{amsfonts}
\usepackage{tikz}
\usepackage[plain]{algorithm}
\usepackage{algpseudocode}
\usepackage{mathtools}
\usepackage{graphicx}
\graphicspath{ {./Images/} }

\DeclarePairedDelimiter\abs{\lvert}{\rvert}%
\DeclarePairedDelimiter\norm{\lVert}{\rVert}%

\makeatletter
\let\oldabs\abs
\def\abs{\@ifstar{\oldabs}{\oldabs*}}
%
\let\oldnorm\norm
\def\norm{\@ifstar{\oldnorm}{\oldnorm*}}
\makeatother

\newcommand*{\Value}{\frac{1}{2}x^2}%

\usetikzlibrary{automata,positioning}

%
% Basic Document Settings
%

\topmargin=-0.45in
\evensidemargin=0in
\oddsidemargin=0in
\textwidth=6.5in
\textheight=9.0in
\headsep=0.25in

\linespread{1.1}

\pagestyle{fancy}
\lhead{\hmwkAuthorName}
\chead{\hmwkClass\ (\hmwkClassInstructor\ \hmwkClassTime): \hmwkTitle}
\rhead{\firstxmark}
\lfoot{\lastxmark}
\cfoot{\thepage}

\renewcommand\headrulewidth{0.4pt}
\renewcommand\footrulewidth{0.4pt}

\setlength\parindent{0pt}

%
% Create Problem Sections
%

\newcommand{\enterProblemHeader}[1]{
    \nobreak\extramarks{}{Problem \arabic{#1} continued on next page\ldots}\nobreak{}
    \nobreak\extramarks{Problem \arabic{#1} (continued)}{Problem \arabic{#1} continued on next page\ldots}\nobreak{}
}

\newcommand{\exitProblemHeader}[1]{
    \nobreak\extramarks{Problem \arabic{#1} (continued)}{Problem \arabic{#1} continued on next page\ldots}\nobreak{}
    \stepcounter{#1}
    \nobreak\extramarks{Problem \arabic{#1}}{}\nobreak{}
}

\setcounter{secnumdepth}{0}
\newcounter{partCounter}
\newcounter{homeworkProblemCounter}
\setcounter{homeworkProblemCounter}{1}
\nobreak\extramarks{Problem \arabic{homeworkProblemCounter}}{}\nobreak{}

%
% Homework Problem Environment
%
% This environment takes an optional argument. When given, it will adjust the
% problem counter. This is useful for when the problems given for your
% assignment aren't sequential. See the last 3 problems of this template for an
% example.
%
\newenvironment{homeworkProblem}[1][-1]{
    \ifnum#1>0
        \setcounter{homeworkProblemCounter}{#1}
    \fi
    \section{Problem \arabic{homeworkProblemCounter}}
    \setcounter{partCounter}{1}
    \enterProblemHeader{homeworkProblemCounter}
}{
    \exitProblemHeader{homeworkProblemCounter}
}

%
% Homework Details
%   - Title
%   - Due date
%   - Class
%   - Section/Time
%   - Instructor
%   - Author
%

\newcommand{\hmwkTitle}{Homework\ \#2}
\newcommand{\hmwkDueDate}{January 28, 2020}
\newcommand{\hmwkClass}{Physics 926}
\newcommand{\hmwkClassTime}{}
\newcommand{\hmwkClassInstructor}{Professor Ken Bloom}
\newcommand{\hmwkAuthorName}{\textbf{Robert Tabb}}

%
% Title Page
%

\title{
    \vspace{2in}
    \textmd{\textbf{\hmwkClass:\ \hmwkTitle}}\\
    \normalsize\vspace{0.1in}\small{Due\ on\ \hmwkDueDate\ at 5pm}\\
    \vspace{0.1in}\large{\textit{\hmwkClassInstructor\ \hmwkClassTime}}
    \vspace{3in}
}

\author{\hmwkAuthorName}
\date{}

\renewcommand{\part}[1]{\textbf{\large Part \Alph{partCounter}}\stepcounter{partCounter}\\}

%
% Various Helper Commands
%

% Useful for algorithms
\newcommand{\alg}[1]{\textsc{\bfseries \footnotesize #1}}

% For derivatives
\newcommand{\deriv}[1]{\frac{\mathrm{d}}{\mathrm{d}x} (#1)}

% For partial derivatives
\newcommand{\pderiv}[2]{\frac{\partial}{\partial #1} (#2)}

% Integral dx
\newcommand{\dx}{\mathrm{d}x}

% Alias for the Solution section header
\newcommand{\solution}{\textbf{\large Solution}}

% Probability commands: Expectation, Variance, Covariance, Bias
\newcommand{\E}{\mathrm{E}}
\newcommand{\Var}{\mathrm{Var}}
\newcommand{\Cov}{\mathrm{Cov}}
\newcommand{\Bias}{\mathrm{Bias}}

\begin{document}

\maketitle

\pagebreak

\begin{homeworkProblem}
	Cosmic-ray muons are produced in the upper atmosphere at an altitude of about 800 m, and then travel toward the earth at nearly the speed of light (0.998c).
	\\
	\\
	(a) The lifetime of the muon is $2.2\times 10^{-6}s$. According to classical physics, how far would the muon travel before decaying? Does it ever reach ground level?
	\\
	\\
	(b) Now answer the same question using special relativity.
	\\
	\\
	(c) from the perspective of the muon at rest in its own frame, its lifetime is $2.2\times 10^{-6}s$. How does it make it to the ground?
	\\
	\\
	(d) Pions are also produced in the upper atmosphere, and the pion lifetime is about a factor of a hundred shorter than the muon lifetime. Do they ever reach the ground?
	\\
	\\
	\textbf{Solution}
	\\
	\\
	(a) 
	\[
		\begin{split}
		v_\mu=&0.008c\\
		=&3\times 10^8 m/s\\
		\tau_\mu=&2.2\times 10^{-6}s
		\end{split}
	\]
	The distance traveled classically is: \(d=v_\mu \tau_\mu=(3\times 10^8 m/s)(2.2\times 10^{-6}s)\)
	\[
		d=660 m
	\]
	Sine the muons are produced at a height of 8000 meters, they don't even get close to the ground classically.
	\\
	\\
	(b) Let's look at this using time dilation. For an observer on the ground, time for the muon would be dilated like: \(\Delta t' =\gamma\Delta t\). In our case with $\beta=0.998$:
	\[
		\gamma=\frac{1}{\sqrt{1-0.998^2}}=15.8
	\]
	So time passes 15.8 times more slowly for the muon. The lifetime then becomes:
	\[
		\tau_\mu'=(2.2\times 10^{-6}s)(15.8)=3.5\times 10^{-5}s
	\]
	In this time period the muon would travel a distance:
	\[
		(d = 3.5\times 10^{-5}s)(3\times 10^8 m/s)=10440m
	\]
	The muon would reach the ground.
	\\
	\\
	(c) The muon would observe length contraction:
	\[
		\Delta z'= \frac{1}{\gamma}\Delta z=505.7m
	\]
	This is less than the 660 meters it can travel in $2.2\times 10^{-6}s$
	\\
	\\
	(d) For the pion, $\tau_\pi=2.2\times 10^{-8}s$
	\[
		\begin{split}
		\Rightarrow \Delta t' =& \gamma\Delta t = (15.82)(2.2\times 10^{-8}s)\\
		=&3.5\times 10^{-7}s
		\end{split}
	\]

\end{homeworkProblem}

\pagebreak

\begin{homeworkProblem}
	Particle A of energy E hits particle B at rest and produces particles $C_1,C_2,...,C_n$.
	\\
	\\
	(a) What is the minimum energy E required for this reaction to take place?
	\\
	\\
	(b) A beam of protons is incident on a target of stationary protons. What is the minimum beam energy needed to create the final state $p\Sigma^+K^0$?
	\\
	\\
	\textbf{Solution}
	\\
	\\
	(a) The energy required to produce $\sum_{i}C_i$ is:
	\[
		E_{CM}=\sum_{i}m_i
	\]
	Where $m_i$ is the mass of state i.
	\\
	\\
	Calculate the center of mass energy calculated before the collision:
	\[
		\begin{split}
		E_{CM}=&\sqrt{s}\\
		P_A^\mu=&(E_A,\vec{p}_A) \\
		P_B^\mu=&(m_B,\vec{0}) \\
		\Rightarrow s =& (P_A^\mu+P_B^\mu)^2\\
		=& P_A^\mu P_{A,\mu}+P_B^\mu P_{B,\mu}+2P_A^\mu P_{B,\mu} \\
		=& m_A^2+m_B^2+2E_A m_B \\
		\Rightarrow E_{CM}=&\sqrt{m_A^2+m_B^2+2E_A m_B} \\
		E_A =& \frac{1}{2m_B}[E_{CM}^2-m_A^2-m_B^2] \\
		=& \frac{1}{2m_B}[(\sum_{i}m_i)^2-m_A^2-m_B^2]
		\end{split}
	\]
	
		(b)
		\[
			\begin{split}
			m_p=0.938 GeV \\
			m_\Sigma=1.189 GeV\\
			m_K = 0.498 GeV
			\end{split}
		\]
		The rest mass of the final state:
		\[
			\begin{split}
			E_{CM}=&(0.938+1.189+0.498)GeV\\
			=&2.625GeV
			\end{split}
		\]
		Using the final result of part (a) we can easily find the beam energy needed:
		\[
			\begin{split}
			E_{beam} =& \frac{1}{2m_B}[(\sum_{i}m_i)^2-m_A^2-m_B^2] \\
			=& \frac{1}{2(0.938)}[(2.625)^2-2(0.938)^2] \\
			=& 2.735 GeV
			\end{split}
		\]
\end{homeworkProblem}

\pagebreak

\begin{homeworkProblem}
	A particle with mass M decays at rest into particles of mass $m_1$ and $m_2$ with $m_1 > m_2$.
	\\
	\\
	(a) Use conservation of four-momentum to calculate the energy and momentum of the two decay particles in the rest frame of the original particle. Express the momentum in terms of $\mu=m_1+m_2$ and $\delta m=m_1-m_2$. If the kinetic energy of the particle is $T_i-E_i-m_i$, which of the two particles has the larger energy, and which the larger kinetic energy?
	\\
	\\
	(b) Compute these energies, kinetic energies and momenta for the decay $D^{*+}\rightarrow D^0\pi^+$. 
	\\
	\\
	(c) Suppose the $D^{*+}$ has momentum q in the laboratory frame. Find the momentum of the two decay products in the lab as a function of $cos\theta$, the Cm angle of decay with respect to the flight path of the $D^{*+}$. 

\end{homeworkProblem}

\pagebreak

\begin{homeworkProblem}

\end{homeworkProblem}

\pagebreak

\begin{homeworkProblem}
\end{homeworkProblem}


\end{document}
